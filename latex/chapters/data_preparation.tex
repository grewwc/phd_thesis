\chapter{Data Preparation}
  As previously mentioned, we use \textit{Kepler} DR24 data 
  \footnote{ https://exoplanetarchive.ipac.caltech.edu/docs/Kepler\_TCE\_docs.html}
  with labels to train out model. It is a CSV file while other types are also 
  supported. Note that the data contains training labels generated by autovetter which 
  is basically a
  random forest technique. Random forest is also a supervised machine learning algorithm,
  thus it also needs labels to tell the algorithm which one is PC. The labels fed into 
  random forest algorithm are classified by humans. The reason why use deep learning method 
  to redo the classification task is because deep learning performs very well when trained 
  with large amount of data. Therefore, we try to use the human-made and normal machine 
  learning method generated labels to train a deep learning model in order to get better 
  accuracy.

  The labels contain four unique values: 
  PC (Planet Candidates) , AFP (Astronomical False Positive), NTP 
  (Non-transiting Phenomenon) and UNK (Unknown). There are 27 columns in the data including 
  the label column named "av\_training\_set". Each other column represents a property of one 
  TCE and some of the columns are more important for our classification task, such as 
  "tce\_period", "tce\_duration", "tce\_prad", "tce\_depth", etc. Meanings of the some 
  column names are listed in the following table.
  \footnote{https://exoplanetarchive.ipac.caltech.edu/docs/API\_tce\_columns.html}

  \begin{table}[!htp]
    \centering
    \scalebox{0.8}{
    \begin{tabular}{|c|c|} 
      \hline 
      \textbf{Column Name} & \textbf{Definition} \\
      \hline 
      tce\_period & time interval between two consecutive transits in days \\
      \hline 
      tce\_depth & starlight drops in ppm \\
      \hline 
      tce\_prad & the planet radius in Earth Radii \\
      \hline 
    \end{tabular}}
    \centering
    \caption[Descriptions of column names.]{Descriptions of column names. The listed 
      properties are more important for identifying PC.}
    \label{table: difinitions_of_column_names}
  \end{table}
  
  The CSV table needs to be preprocessed. First of all, we need to drop all data labeled with 
  'UNK'. Then since deep learning algorithms only deal with numerical labels, we need to 
  label PC and non-PC as 1 and 0 respectively. Non-PC includes both AFP and NTP because we do 
  binary classification to find out planet candidates. After processing the basic properties 
  for each TCE, we also need to download their light curves.

  \section{Download Light Curves}
    Light curves are downloaded based on the column name "kepid" in the \textit{Kepler} DR24 
    data table previously discussed. There are about 16 fits files which is composed of four 
    year observation corresponding to a kepid. All the light curves can be found on the 
    website \blackhref{https://archive.stsci.edu/pub/kepler/lightcurves/0020/}
    {https://archive.stsci.edu/pub/kepler/lightcurves/0020/}. A tricky thing is that 
    there are 20367 TCE
    in the table containing more than 330,000 light curve files. Downloading these 
    files can be very time consuming (about 2 weeks by estimation). Therefore, I scrape the 
    webpage in parallel to filter the light curves data and download them. 

    Take kepid 000757137 as an example. The URL of the light curves for this TCE is 
    \blackhref{http://archive.stsci.edu/pub/kepler/lightcurves/0007/000757137}
    {http://archive.stsci.edu/pub/kepler/lightcurves/0007/000757137}. There are 17 light 
    curve files as the following screen shot shown.
    \begin{figure}[!htp]
      \centering
      \includegraphics[scale=0.4]{kepid_light_curves_example.PNG}
      \caption{The screen shot of webpage containing light curves of kepid 000757137}
      \label{fig: kepid_light_curves_example}
    \end{figure}

    We can find the hyper link from the anchor tag and navigate to the URL of the desired 
    data. The code is written in Golang, which is very good at concurrency. All code can be 
    found on \blackhref{https://github.com/grewwc/exoplanet}{github}.
    In short, the basic procedure is 
    \begin{enumerate}
      \item Parse all the hyperlinks from the root URL: \\
        \blackhref{http://archive.stsci.edu/pub/kepler/lightcurves}
        {http://archive.stsci.edu/pub/kepler/lightcurves}. Join the current URL with 
        the parsed hyperlinks to generate new URLs. 
      \item Launch new goroutine to search from the newly generated URLs.
      \item Get the hyper links from the child URLs. If the hyper links are fits files, 
        then download them to a local directory in parallel. 
        Otherwise, as the procedure 1, get new 
        URLs and launch new goroutine to do the search recursively.
    \end{enumerate}

    Another problem is if I search from too many URL simultaneously, my IP will be blocked
    temporarily. Therefore, I need to control the total amount of processes. After some 
    attempts, I find that 200 processes are good enough and all light curves can be 
    downloaded within 3 hours. The point is data are usually very large and complex 
    in the field of astronomy, thus dealing with data efficiently is important.

    
  \section{Light Curves Pre-processing}
    As previously discussed, the depths of TCE are in general very small and the median 
    value is around 125 ppm. The following figure is the four-year observation data. 
    Thus we need to fold the light curves and bin them in order to magnify the effect.
    Moreover, we can see some data points in the top right side of the figure. We also 
    need to remove the outliers to improve out model's prediction accuracy.
    
    \begin{figure}[!htp]
      \centering
      \includegraphics[scale=0.6]{11442793_time_flux.png}
      \caption[Plot of normalized flux to time of kepid 11442793.]
        {Plot of normalized flux to time of kepid 11442793. The flux is normalized by 
        dividing the original flux by their median value.}
      \label{fig: 11442793_time_flux}
    \end{figure}

    In addition, there is long-term trend which is clearly not a transit signal
    in the light curves for some kepids. For example, we can see a clear periodical 
    fluctuation in the light curve for kepid 1164109 as the following figure shown.
    The fluctuation has bad effects for the deep learning model, thus even though we don't 
    have to deal with the long-term trend, removing the trend may increase the model's 
    performance.

    \begin{figure}[!htp]
      \centering
      \includegraphics[scale=0.5]{1164109_time_flux.png}
      \caption[Plot of normalized flux to time of kepid 1164109.]
        {Plot of normalized flux to time of kepid 1164109.}
      \label{fig: 1164109_time_flux}
    \end{figure}

    Finally, since there may be more than one planet associated with a kepid, we remove 
    signals of other planets in the same system to avoid interference. The complete 
    progress of light curves preprocessing is described as follows.

    \subsection{Remove Multiple Signals in the Same Stellar System}
      The following table shows some useful parameters of TCE for kepid 1162345. 
      When preprocessing kepid 1162345, we fetch all the TCE --- there are 3 TCE in the stellar 
      system. Then we process the TCE one by one, with other two TCE removed. The remove logic is simple:
      mask out the data points within the range [center-duration, center+duration] and return the new 
      flux and time pairs. 

      \begin{table}[!htp]
        \centering
        \scalebox{0.8}{
        \begin{tabular}{|c|c|c|c|} 
          \hline 
          \textbf{tce\_plnt\_number} & \textbf{tce\_period (day)} & \textbf{tce\_duration (hr)} & \textbf{tce\_depth (ppm)} \\
          \hline 
          2	& 0.831850 & 2.392 & 2.636\\
          \hline 
          3	& 0.831833 & 2.181 & 27.100\\
          \hline 
          1	& 0.831777 & 2.349 & 24.270	\\ 
          \hline 
        \end{tabular}}
        \centering
        \caption[Parameters of three TCE for kepid 1162345.]
          {Parameters of three TCE for kepid 1162345. When processing the TCE for planet 2, we remove the signals for 
          planet 1 and 3.}
        \label{table: params_1162345}
      \end{table}
    