
\chapter{Introduction}
  \section{Transit Photometry}
    A planet is too faint to be found directly by telescopes, so people study it 
    by observing its 
    host star. Since planets orbit around their host stars, they have some periodical effects 
    on the stars such as gravitational pull and light blocking. When a exoplanet pass in front 
    of its host star, it blocks a very small fraction of the star. By measuring the drops of  
    the starlight, basic stats of the planet can be obtained such as period and size. 

    Because planets are in general very tiny compared to host stars, the drops of the 
    starlight are very small. Therefore, some statistic methods are used to identify 
    transits. The confirmed transits that passed statistic tests are called TCE 
    (Threshold-Crossing Events).
    In \textit{Kepler} DR24 data products (the data used in this thesis), 
    the mean and median of TCE are 
    about $12516$ ppm and $125$ ppm respectively (ppm stands for Parts Per Million). The 
    distribution of TCE depths is shown in the following figures.

    \begin{figure}[!htp]
      \centering
      \includegraphics[scale=0.7]{tce_depth_hist.png}
      \caption[The distribution of TCE depths of \textit{Kepler} DR24 data.]
      {The distribution of TCE depths of \textit{Kepler} DR24 data. There are 20367 TCE
      in the data and only 1831 are larger than 10000 ppm (less than 10\%). 
      The histogram shows that most of TCE depths are smaller than 500 ppm.}
    \label{fig: tce_hist_and_pie} 
    \end{figure}

  
  \section{\textit{Kepler Space Telescope}}
    \textit{Kepler Space Telescope} is a space telescope for finding
    Earth-like terrestrial exoplanets. It was launched on March 7th, 2009 and 
    was retired in late 2018. During its service, \textit{Kepler Space Telescope} observes
    more than 500 thousand stars and finds about 2600 exoplanets.
    
    The \textit{Kepler} pipeline is a set of programming tools which can generate
    calibrated light curves which can then be fed into the
    algorithms to discriminate between planet candidates (PC) and other types of light curves.
    Threshold Crossing Events (TCE) are generated
    after Transiting Planet Search (TPS) which is part of 
    the \textit{Kepler} pipeline. And then we can identify planet candidates
    by analyzing the generated TCE. 

    \textit{Kepler} team developed a tool called \textit{Kepler Robovetter} 
    \footnote{https://github.com/nasa/kepler-robovetter}
    to identify PCs and false positives and it achieves a high accuracy (over 97\%).
    It is a traditional algorithm which is purely written in C++ and 
    is very fast. However, it is difficult to understand
    and one can view its source code on the github. Therefore, try different method like 
    machine learning may be a feasible choice.
  
  \section{Machine Learning}
      Machine learning is a subset of artificial intelligence and has been successfully used 
      in many areas for a variety of tasks such as self-driving cars and optical character
      recognition (ORC) which can extract words and their meanings from images. Machine 
      learning methods can even generate art works such as pictures which are nearly 
      indistinguishable from art works created by artists. Therefore, it is natural to think 
      that machine learning methods can also be applied to astronomy. 

      Machine learning can be classified differently based on different criteria. For example,
      if we predict a product's price which is a concrete value, we are doing a regression task. 
      Otherwise, if we need to classify pictures as dogs or cats, then it is a classification 
      task. Meanwhile, an algorithm is called supervised learning if we train the machine 
      learning model by giving corresponding labels. Otherwise, it is a 
      non-supervised learning algorithm. In this thesis, we input the TCE with 
      corresponding category labels generated by 
      \textit{Kepler} pipeline and output a number 0 or 1 to indicate PC or non-PC. Thus,
      we use a supervised learning method to do a classification task.

      There are a large quantity of machine learning algorithms for classification tasks
      such as logistic regression, decision trees and support vector machines. However, 
      in this thesis, besides trying these machine learning methods, we 
      focus more on deep learning methods, especially convolution neural networks (CNN).


  