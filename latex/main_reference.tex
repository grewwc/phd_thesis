\documentclass[a4paper, 12pt]{report}
\usepackage{geometry}	
\usepackage[utf8]{inputenc}
\usepackage{amsmath}
\usepackage{multicol}
\usepackage{titlesec}
\usepackage{graphicx}
\usepackage{wrapfig}
\usepackage{textcomp}
\usepackage{caption}
\usepackage{subcaption}
\usepackage{comment}
\usepackage{etoolbox}
\usepackage{anyfontsize}
\usepackage{url}
\usepackage{multirow}
\usepackage{array}
\usepackage{tabu}
% \usepackage{hyperref}
\usepackage[colorlinks,citecolor=black, urlcolor=black]{hyperref}
\usepackage{color}
\usepackage{epigraph}
\usepackage{makebox}
\usepackage{graphicx}
\usepackage{array}
\usepackage{setspace}
% \usepackage{natbib}

\usepackage{mathrsfs}
\usepackage{caption}
\usepackage{float}
\usepackage{makecell}
\captionsetup{font={stretch=1.2}}
\usepackage[english]{babel}
% \graphicspath{{C:/Users/User/Desktop/Thesis/}{/Users/grewwc/Desktop/Thesis/}}

\usepackage{biblatex}
\renewbibmacro{in:}{}
\addbibresource{bibfile.bib}
\DeclareFieldFormat{pages}{#1}% no p
\DeclareFieldFormat{journaltitle}{\mkbibemph{#1}\isdot}

\renewbibmacro*{journal+issuetitle}{%
  \usebibmacro{journal}%
  \setunit*{\addcomma\space}%
  \iffieldundef{series}
    {}
    {\newunit
     \printfield{series}%
     \setunit{\addspace}}%
  \usebibmacro{volume+number+eid}%
  \setunit{\addspace}%
  \usebibmacro{issue+date}%
  \setunit{\addcolon\space}%
  \usebibmacro{issue}%
  \newunit}

\renewbibmacro{in:}{\addcomma\addspace}
\DeclareFieldFormat[article]{title}{``#1''}
\DeclareFieldFormat[unpublished]{title}{``#1''}
\DeclareFieldFormat[book]{title}{``#1''}
\DeclareFieldFormat[article]{volume}{#1}
\DeclareFieldFormat[article]{number}{\bibstring{number}~#1}
\DeclareFieldFormat{pages}{#1}
\renewbibmacro*{volume+number+eid}{%
  \printfield{volume}%
  \setunit*{\addcomma\addspace}%
   \printfield{number}%
   \setunit{\addcomma\space}%
   \printfield{eid}}
\renewcommand{\arraystretch}{2}
% \renewcommand{\bibname}{REFERENCES}
\hypersetup{
    colorlinks,
    citecolor=black,
    filecolor=black,
    %linkcolor=[RGB]{0,204,0},
    linkcolor=black,
    urlcolor=black
}

% \captionsetup[figure]{labelfont=bf, font=footnotesize}

\setlength{\parskip}{1em}
% \setcounter{tocdepth}{4} 
% \setcounter{secnumdepth}{4}

%this several lines is for: no number before suction. (This is a bug)
\makeatletter
\patchcmd{\ttlh@hang}{\parindent\z@}{\parindent\z@\leavevmode}{}{}
\patchcmd{\ttlh@hang}{\noindent}{}{}{}
\makeatother


\geometry{
	a4paper,
	% total={210mm,297mm},
 	left=30mm,
 	top=30mm,
  right=30mm,
  bottom=30mm,
}

% \title{\textbf{Broadband Spectra Analysis of Three Energetic Millisecond Pulsars}\\ \vspace{1cm}
% 			{\large Department of Physics, The University of Hong Kong, Pokfulam Road, Hong Kong}\\ \vspace{1cm}
% 			% {\includegraphics[scale=0.2]{{hku.png}}}\\ \vspace{3cm}
% }

\title{\textbf{Gamma-ray Spectral Analysis of Three Energetic Millisecond Pulsars} \\ \vspace{1cm}
  {\large Department of Physics, The University of Hong Kong, Pokfulam Road, Hong Kong}}

\date{}
\author{Wang Wenchao  \\3030053350}
\setlength{\columnsep}{1cm}


\newcommand{\mycaption}[1]{\protect \caption{#1}}
%Below is the main content.

%insert a single figure.
%parameters: 
%%  1. path
%%  2. scale
%%  3. caption
\newcommand{\singleFig}[3]{
  \begin{figure}[!htp]
    \centering
    \includegraphics[scale=#2]{#1}
    \caption{#3}
    \label{fig: #1}
  \end{figure}
}

\newcommand{\gj}[0]{
  Goldreich-Julian charge density
}

\newcommand{\fgl}[0]{
  \textit{Fermi} LAT four-year Point Source Catalog
}

\newcommand{\question}[1]{
  $<$\textbf{question}$>$#1$<$\textbf{/question}$>$
}

\newcommand{\change}[1]{
  $<$\colorbox{red}{\textbf{change}}$>$#1$<$\colorbox{red}{\textbf{/change}}$>$
}

\newcommand{\add}[1]{
  $<$\colorbox{red}{\textbf{add}}$>$#1$<$\colorbox{red}{\textbf{/add}}$>$
}

\newcommand{\mayAdd}[1]{
  $<$\colorbox{red}{\textbf{mayAdd}}$>$#1$<$\colorbox{red}{\textbf{/mayAdd}}$>$
}


\newcommand{\mayChange}[1]{
  $<$\colorbox{red}{\textbf{mayChange}}$>$#1$<$\colorbox{red}{\textbf{/mayChange}}$>$
}

\newcommand{\myComment}[1]{
  %#1 
  \newline
}

\newcommand{\Notice}[1]{
  $<$\textbf{Notice}$>$#1$<$\textbf{/Notice}$>$
}

\newcommand{\blackhref}[2]{
  \href{#1}{\color{black}{\textit{\small #2}}}
}



\begin{document}

\begin{titlepage}
  \begin{center}
  \vspace*{3cm}
  % Gamma-ray Spectral Analysis of Three Energetic Millisecond Pulsars
  \Large \textbf{GAMMA-RAY SPECTRAL ANALYSIS 
    \\[0.5cm] OF THREE ENERGETIC \\ [0.5cm] 
    MILLISECOND PULSARS \\[3.7cm]}
  \small
    A THESIS \\[0.5cm] 
    SUBMITTED TO THE DEPARTMENT OF PHYSICS \\[0.5cm]
    OF THE UNIVERSITY OF HONG KONG \\[0.5cm]
    IN PARTIAL FULFILLMENT OF THE REQUIREMENTS \\[0.5cm]
    FOR THE DEGREE OF \\[0.5cm]
    MASTER OF PHILOSOPHY \\[4.2cm]

    \large By \\[0.5cm]
    \large Wenchao Wang \\[0.5cm]
    August 2018
    \newpage 
    \centering 
    Abstract of thesis entitled \\[1.2cm]
    \Large \textbf{GAMMA-RAY SPECTRAL ANALYSIS 
    \\[0.5cm] OF THREE ENERGETIC \\ [0.5cm] 
    MILLISECOND PULSARS \\[0.5cm]}
    \normalsize Submitted by \\ [0.5cm]
    \Large Wenchao Wang \\[1cm]
    \normalsize for the degree of Master of Philosophy \\[0.5cm]
    at the University of Hong Kong \\[0.5cm]
    in August 2018 \\[1.9cm]
  \end{center}
    \doublespacing
    \noindent A millisecond pulsar (MSP) is a fast-spinning pulsar whose rotational period is a few 
    milliseconds. MSPs are believed to be old pulsars spun up by their companion stars. PSRs
    J0218+4232, B1821$-$24 and B1937+21 are among the most energetic and fastest-spinning  
    MSPs. They have been studied in radio, X-ray and gamma-ray bands, and show 
    aligned pulse profile in different energy bands. However, all previous gamma-ray studies 
    were done with previous \textit{Fermi} LAT Pass 7 data or earlier. The \textit{Fermi} LAT Pass 8 data was 
    published in 2015 and has substantial improvements, such as increased effective area and 
    wider energy range. Since the recent gamma-ray spectral analyses of the three MSPs are 
    relatively old, I re-analyzed the gamma-ray spectra of the three MSPs with four-year more 
    \textit{Fermi} LAT observational data and newly published Pass 8 data. Additionally, new X-ray 
    studies of the three MSPs using NuSTAR had been published. I obtained better fit results
    for gamma-ray spectra of the three MSPs with smaller error bars and larger test statistic 
    values. I built a numerical model to explain the high-energy emission from X-rays to 
    gamma-rays based on a two-layer outer gap model. By minimizing the differences between 
    the predictions of the two-layer model and the real data, I fitted three independent 
    parameters of the model. It is found that the simplified two-layer model can predict 
    broadband spectra of the three MSPs which are very close to the observational data from 
    in both X-rays and gamma-rays.

\end{titlepage}

% \bibpunct{(}{)}{;}{a}{}{,}
% \bibliographystyle{apj}

\linespread{1.25} 
\maketitle

\pagenumbering{roman}


% \thispagestyle{empty}
\cleardoublepage
\chapter*{Declaration}
  \doublespacing 
  I hereby declare that this whole dissertation report is my own work, except the parts with due
  acknowledgment, and that it has not been previously included in a thesis, dissertation or
  report submitted to this University or to any other institution for a degree, diploma or other
  qualifications. \\ \vspace{3cm}

  \hspace{7cm} Signature: \rule{3cm}{0.4pt} 

  \hspace*{7.5cm} Name:  Wenchao Wang 

  \hspace*{7.6cm} Date:  August 2018 

\addcontentsline{toc}{chapter}{Declaration}

\cleardoublepage
\chapter*{Acknowledgments}
  \doublespacing
  Lots of people give me much precious help. First of all, I really appreciate my 
  supervisor Dr. Stephen Chi Yung Ng for his great support, patience and kindness. Prof. Takata 
  also provides me much help in term of pulsars' emission mechanisms and numerical simulations and 
  is really charming. In addition, I thank Ms. Ruby Cho Wing Ng for her invaluable guidance and 
  advice. They give me not only knowledge and techniques but also encouragements. Therefore, I 
  would like to express my most sincere thanks to them. 
\addcontentsline{toc}{chapter}{Acknowledgments}
 

% \cleardoublepage
% \chapter*{Abstract}
%   \doublespacing
%   PSRs J0218+4232, B1821$-$24 and B1937+21 are among the most energetic and fastest-spinning 
%   millisecond pulsars (MSPs). They have been studied in all radio, X-ray and gamma-ray bands. 
%   The \textit{Fermi} LAT Pass 8 data was published in 2015 and has lots of advantages over 
%   the old Pass 7 data, such as increased effective area and wider energy range. Since 
%   the recent gamma-ray spectral analysis of the three MSPs are relatively old, 
%   I redo the gamma-ray spectral analysis of the MSPs with 
%   four-year more \textit{Fermi} LAT observational data and newly published Pass 8 data. 
%   I obtain better fit results for gamma-ray spectra of the three MSPs with smaller errors 
%   and larger test statistic values. I also do numerical simulations to test the 
%   two-layer model using the new observational data. By minimizing the differences between 
%   the predictions of the two-layer model and the real data, I fit the independent 
%   parameters of the two-layer model. I find that the simplified two-layer model can 
%   predict broadband spectra of the three MSPs which are very close to the observational data 
%   from \textit{Fermi} LAT in most energy ranges.

% \addcontentsline{toc}{chapter}{Abstract}

\tableofcontents

\cleardoublepage
\phantomsection
\addcontentsline{toc}{chapter}{\listfigurename}
\listoffigures
% \newpage

\cleardoublepage
\phantomsection
\addcontentsline{toc}{chapter}{\listtablename}
\listoftables
% \newpage

\pagenumbering{arabic}


 
\cleardoublepage
\newpage
\listoftables


\chapter{Introduction}   	   
  \section{Neutron Stars and Pulsars}
    Neutron stars are produced by supernovae explosion of massive stars whose mass 
    is above eight solar mass. After a supernova explosion, a star leaves a central region. 
    And the central region collapses because of the effect of 
    gravity until protons and electrons combine to form neutrons 
    ($e^{-}+p\rightarrow n+\nu_{e}$) ---the reason why they are called ``neutron stars''.  
    Because neutrons have no electromagnetic force on each other, they can be squeezed very 
    tightly. Therefore, a neutron star has a tremendous high density 
    (about $5\times 10^{17} \mbox{kg/m}^3$) and its diameter and mass are about 20km and 
    1.4 solar mass respectively. What
    prevents a neutron star to continue to contract is the degeneracy pressure of neutrons. 
    
    Pulsars are fast-spinning neutron stars. They have rotational periods from a few 
    milliseconds to several seconds. For example, the rotational period of PSR B1937+21 is 
    about 1.56ms while PSR B1919+21 is approximately 1.34s. As we know, a star can be ripped 
    by centrifugal force if the star rotates too fast. We can estimate lower limits of 
    density of a star with the equation $\rho=\frac{3\pi}{P^2G}$, where $P$ is the 
    rotational period of a pulsar. Just for simplicity, let $P$ be 1s. Then we 
    get $\rho\approx 1.4\times 10^{11}\mbox{kg/m}^3$. With the knowledge that the 
    density of a white dwarf is about $1\times 10^9\mbox{kg/m}^3$ which is smaller than 
    the lower density limit, the observed fast-spinning stars belong to the kind of stars 
    which are much denser than white dwarfs. As a result, neutron stars are ideal candidates 
    for pulsars. 
    
    More than 2000 pulsars have been discovered so far. Most of them are in the disk of 
    our Galaxy while we also can find a small portion of them in high latitude, which can be 
    seen clearly in Figure \ref{fig: spatial_distribution}. This may 
    because they cannot escape the gravitational potential if their kinetic energy is not 
    large enough. Besides, even though they have large enough velocities to escape from their 
    birth region, there are some probabilities that they become nearly non-detectable before 
    reaching high latitude. 

    \begin{figure}[!htp]
      \centering
      \includegraphics[scale=0.35]{pulsar_distribution.png}
      \caption{Spatial distribution of some pulsars in galactic coordinate system.}
      \label{fig: spatial_distribution}
    \end{figure}

    
  \section{Emission Mechanism of Pulsars}
    Although the emission mechanism of pulsars has not been fully understood yet, some 
    models have been developed 
    trying to explain observational data. The following is a simple model that can explain 
    some basic features of pulsars spectra. I will first introduce the magnetic dipole model, 
    then the synchrotron radiation and inverse Compton radiation. 
    \subsection{Magnetic Dipole Model}
      Assuming a pulsar has a magnetic dipole moment $\vec{m}$, the angle between rotation axis and 
      direction of 
      $\vec{m}$ is $\alpha$, its angular velocity is $\Omega$, radius is R and moment of inertia is $I$. 
      Also assuming that energy of electromagnetic radiation is completely from the rotational energy, 
      its spin-down rate can be written as: 
      $$
          \dot{\Omega}=-\frac{B_p^2 R^6 \Omega^3 \sin{\alpha}^3}{6c^3I}
      $$
      where $B_p$ is magnetic field strength in the pole of the pulsar. Its surface magnetic field can 
      also be estimated by:
      $$
          B_s=\sqrt{\frac{3c^3I}{2\pi^2R^6}P\dot{P}}=3.2\times 10^{19}\sqrt{P\dot{P}}
      $$
      where $B_s$ is the strength of the surface magnetic field. \\
      \indent In general, a pulsar's spin-down rate can be expressed as $\dot{\Omega}=-K\Omega^{n}$, 
      where K is a constant and n is called braking index. In magnetic dipole model n is 
      three \cite{Tong2015}. Then the characteristic age of the pulsar can be 
      defined as $P/2\dot{P}$ in the magnetic dipole model. For example, the Crab 
      pulsar's rotation period is about $0.033s$ and period derivative is 
      $4.22\times 10^{-13}s/s$. The characteristic age is about 1200 years. The pulsar 
      is a remnant of a supernova which is observed by ancient astronomers in 1054 
      AD, so the record shows that characteristic age can give us and an order of magnitude 
      estimation of a pulsar's real age. \\
      \indent 
      Although the braking index is 3 in the magnetic dipole model, most of the pulsars' 
      braking index is less than 3 as shown in Figure \ref{fig:braking_index}. The reason is 
      that if a pulsar's spin down is completely determined by pulsar winds, the braking 
      index is one. Thus, the real braking index should be a combination of 1 and 3, which is usually 
      less than three \cite{PhysRevD.91.063007}.
        
      \begin{figure}[!ht]
        \centering
        \includegraphics[scale=0.6]{table.png}
        \caption{Braking indices of some pulsars.}
        \label{fig:braking_index}
      \end{figure}
  
    \subsection{Synchrotron Radiation}
      Synchrotron radiation is a special case of cyclotron radiation when a particle's
      speed is comparable to the speed of light. Because of the relativistic beaming 
      effect, a very short radiation pulse can be observed when speeds of particles are
      large. I only aim to analyze the spectral properties of MSPs, so I focus on the 
      spectrum property of synchrotron radiation. 
      With Larmor's Formula we can derive the synchrotron radiation power of an electron:
      \begin{equation}
        \label{func: sync_total_power_precise}
        P = \frac{2e^4\gamma^2\beta^2B_{\perp}^2}{3m_e^2c^3} ,
      \end{equation}
      where $\gamma$ is the Lorentz factor of the electron, $\beta=v/c$ and $B_{\perp}$ 
      is the strength of magnetic field perpendicular to the circular motion plane. 
      When $\beta \sim 1$, Function \ref{func: sync_total_power_precise} can be 
      simplified as:
      \begin{equation}
        \label{func: sync_total_power_simplified}
        P = \frac{2}{3}\frac{e^2c}{R^2}\gamma^4 ,
      \end{equation}
      where $R = E / e B_{\perp}$ is the radius of the electron's circular motion. 
      Furthermore, the power spectrum of a single electron  
      can be described by Function \ref{func: syncrothron spectrum}
      \begin{eqnarray}
        \label{func: syncrothron spectrum}
        P\left(\nu\right) &=& \frac{\sqrt{3} e^3 B \sin{\alpha}}{m c^2} 
          \left(\frac{\nu}{\nu_c}\right) \int_{\nu / \nu_c}^{\infty} K_{5/3}\left(\eta \right)d\eta  \nonumber \\
          &=& \frac{\sqrt{3}e^2}{m_eR}\gamma \left(\frac{\nu}{\nu_c}\right) \int_{\nu / \nu_c}^{\infty} K_{5/3}\left(\eta \right)d\eta ,
      \end{eqnarray}
      where $\nu_c$ is the critical frequency and $K_{5/3}$ is modified Bessel function. 
      The critical frequency can be expressed by Function \ref{func: critical_frequency}
      \begin{eqnarray}
        \label{func: critical_frequency}
        \nu_c &=& \frac{3}{2} \gamma^2 \nu_{cyc} \sin{\alpha} \nonumber \\
              &=& \frac{3}{4\pi} \frac{c}{R} \gamma^3 ,
      \end{eqnarray} 
      where $\alpha$ is the pitch angle and the $\nu_{cyc}$ is the frequency of 
      corresponding cyclotron radiation. These functions do not give us very much 
      information because of the integration of the modified Bessel function. We let 
      $x = \nu / \nu_c$ and fix the environment variables such as magnetic field ($B$), 
      Function \ref{func: syncrothron spectrum} becomes: 
      \begin{equation}
        \label{func: to_x}
        P\left(\nu\right) = C \times x \int_{x}^{\infty} K_{5/3}\left(\eta \right)d\eta ,
      \end{equation}
      where $C$ is a constant dependent on $B$. Thus, in order to analyze the power spectrum of synchrotron radiation,
      we only concentrate on the later part, which is
      \begin{equation}
        \label{func: fx}
        F\left(x\right) = x \int_{x}^{\infty} K_{5/3}\left(\eta \right)d\eta .
      \end{equation}
            
      \begin{figure}[!htp]
        \centering 
        \includegraphics[scale=0.5]{sync_spectrum_loglog_combined.png}
        \caption[Spectrum shape of synchrotron radiation for a single particle]
        {Spectrum shape of synchrotron 
        radiation for a single particle (top). According to Equation \ref{func: fx},
        the top figure \ref{func: fx} describes the general shape of power spectrum of 
        synchrotron radiation. When the frequency is larger than
        the critical frequency $\nu_c$, the power goes down dramatically. However, the 
        top figure does not show the information that at what frequency the charged 
        particle emits the strongest power, which is revealed in the bottom figure. 
        The bottom figure shows that most energy is emitted around the critical frequency.}
        \label{fig: sync_spectrum_loglog_combined}
      \end{figure}
      % \singleFig{sync_spectrum_loglog_combined}{0.5}{Spectrum shape of synchrotron 
      % radiation for a single particle (top). According to Equation \ref{func: fx},
      % the top figure \ref{func: fx} describes the general shape of power spectrum of 
      % synchrotron radiation. When the frequency is larger than
      % the critical frequency $\nu_c$, the power goes down dramatically. However, the 
      % top figure does not show the information that at what frequency the charged 
      % particle emits the strongest power, which is revealed in the bottom figure. 
      % The bottom figure shows that most energy is emitted around the critical frequency.}

      In reality, synchrotron radiation is not generated by a single particle. We 
      describe the number density distribution of electrons with respect to energy by a 
      single power-law model:
      \begin{equation}
        \label{func: sync_number_density}
        N\left(E\right) \approx C E^{-\delta} .
      \end{equation}
    
      For simplicity, we set the ambient magnetic field $B$ to be a constant and make an 
      approximation that all radiations are at a single frequency:
      \begin{equation}
        \label{func: sync_approximation}
        \nu \approx \gamma^2 \nu_{cyc} ,
      \end{equation}
      where the meaning of $\nu_{cyc}$ is the same as Function 
      \ref{func: critical_frequency}. Our objective is to know the relationship between 
      total power of all electrons and their radiation frequency. We describe the 
      relationship as Equation \ref{func: sync_power_single_frequency}
        
      \begin{eqnarray}
        \label{func: sync_power_single_frequency}
        -P\left(E\right)N\left(E\right)dE &=& Q_{\nu} d\nu\\
        P\left(E\right) &=& \frac{4}{3} \sigma_{T} \beta^2 \gamma^2 c U_B ,
      \end{eqnarray} 
      where $\sigma_{T}$ is electron Thompson scattering section, $U_B$ is energy 
      density of the environment magnetic field, $Q_{\nu}$ is the emission coefficient 
      of synchrotron radiation and $E=\gamma m_e c^2$. With Equation 
      \ref{func: sync_approximation}, we have
      \begin{equation}
        \label{func: sync_combine}
        P = \frac{dE}{d\nu} = \frac{m_e c^2}{2\sqrt{\nu \nu_{cyc}}} .
      \end{equation}
      Combine Functions \ref{func: sync_combine} and 
      \ref{func: sync_power_single_frequency} we get:
      \begin{equation}
        Q_{\nu} = \frac{4}{3} \sigma_{T} \beta^2 \gamma^2 c U_B \frac{m_e c^2}{2\sqrt{\nu \nu_{cyc}}} N\left(E\right) .
      \end{equation}
      Ignoring constants in Function \ref{func: sync_power_single_frequency} we have 
      \begin{equation}
        \label{func: sync_final}
        Q_{\nu} \propto \nu^{(1-\delta)/2} .
      \end{equation}
      Function \ref{func: sync_final} shows that if the number density of electrons is 
      a power-law distribution, the spectrum of synchrotron radiation is also a 
      power-law model.  
            

    \subsection{Inverse-Compton radiation}
      If an energetic photon collides with a charged particle from a 
      proper incident angle, the photon's energy decreases and its direction changes. 
      This is the process of Compton Scattering. Inverse-Compton radiation is the 
      opposite process and a low energy photon gains energy from an ultra-relativistic 
      electron in the process. 
      
      \singleFig{inverse_compton}{0.45}{Two photons collide with an 
      electron. In the frame $S^{\prime}$, two photons collide with a rest electron 
      successively. In the frame $S$, the electron is no longer at rest and the 
      positions of the two events are $x_1$ and $x_2$}
      As Figure \ref{fig: inverse_compton} shows, in the laboratory frame ($S$), the 
      incident angle and energy of a photon is $\theta$ and $h \nu$ respectively. 
      The speed of the electron is $v$. In the electron rest frame ($S^{\prime}$), 
      we change the denotation to $\theta^{\prime}$ and $h \nu^{\prime}$. Also, let 
      the position of the electron be the origin point of $S^{\prime}$. We can study 
      the whole process in the $S^{\prime}$ frame, then transfer the result to the $S$ 
      frame by Lorentz transformation. 

      In the $S^{\prime}$ frame, the electron is at rest, so its energy is $m_e c^2$. 
      For Inverse Compton scattering, the energy of an incident photon 
      (less than several $keV$) is much less than the rest energy of an electron 
      (about $0.51MeV$) and the relationship can be expressed by 
      $h\nu^{\prime} \ll m_e c^2$. Therefore, this can be treated as Thompson 
      Scattering process. Let the Poynting vector of incident photons be 
      $\vec{S}^{\prime}$ and their energy density be $U_{rad}^{\prime}$, we have 
      equation \ref{eq: poynting_and_energy_density}
      \begin{equation}
        \label{eq: poynting_and_energy_density}
        \vec{S}^{\prime} = c U_{rad}^{\prime} .
      \end{equation}
      The electron absorbs the energy of the incident photons and then is accelerated. 
      Thus, the accelerated electron will emit part of the energy taken from the incoming 
      photon and the power of scattered radiation is denoted as $P^{\prime}$.
      The ratio can be described by Thompson Scattering cross section $\sigma_{T}$ 
      which is:
      \begin{equation}
        \label{eq: thompson_cross_section}
        \sigma_{T} = \frac{8\pi}{3} \left(\frac{e^2}{m_e c^2}\right)^2 ,
      \end{equation}
      and the relationship between the electron radiation power and incoming photon 
      energy flux can be described by Equation \ref{eq: relationship_power_poynting}
      \begin{equation}
        \label{eq: relationship_power_poynting}
        P^{\prime} = \left| \vec{S}^{\prime} \right| \sigma_{T} ,
      \end{equation}
      Combine Equations \ref{eq: poynting_and_energy_density} and 
      \ref{eq: relationship_power_poynting}, the radiation power emitted by the 
      electron is: 
      \begin{equation}
        \label{eq: final_relationship}
        P^{\prime} = c \sigma_{T} U^{\prime}_{rad} .
      \end{equation}

      Then we need to find the relationship between the two frames $S$ and $S^{\prime}$. 
      It mainly contains two parts: the relationship between $P$, $P^{\prime}$ and 
      $U_{rad}$, $U_{rad}^{\prime}$. Since $P = dE/dt$ and it is Lorentz invariant 
      in inertial frame, we get the equation: 
      \begin{equation}
        \label{eq: power_is_equal}
        P = P^{\prime} .
      \end{equation}
      Now we hope to write $U_{rad}^{\prime}$ in terms of $U_{rad}$. $U_{rad}$ is 
      comprised by energy of a single photon and photon density. In the following
      analysis, all the denotations are correspondent to Equation \ref{fig: inverse_compton}. 
      According to the relativistic Doppler shift formula, we have: 
      \begin{equation}
        \label{eq: doppler_shift}
        h \nu^{\prime} = \left(h \nu\right) \gamma \left(1 + \beta \cos{\theta} \right) ,
      \end{equation}
      where $\beta = v / c$ and $\gamma$ is Lorentz factor of an ultra-relativistic 
      electron. Then we calculate the photon density. In the frame $S^{\prime}$, the 
      photon density is inverse proportional to the time interval ($\Delta t$) between 
      the two photons striking the electron since total number of photons is Lorentz 
      invariant. In laboratory frame $S$, we consider two photons collide with the 
      electron at the 4-dimension vector of 
      $\left(x_{1}, 0, 0, t_{1}\right)$ and $\left(x_{2}, 0, 0, t_{2}\right)$. 
      According to the Lorentz transformation between inertial frames: 
      \begin{equation}
        \label{eq: lorentz_transfer_general}
          \begin{cases}
            & x = \gamma \left( x^{\prime} + \beta c t^{\prime} \right)\\
            & y = y^{\prime} \\
            & z = z^{\prime} \\ 
            & t = \gamma \left(t^{\prime} + \frac{\beta x^{\prime}}{c}\right) ,
          \end{cases}       
      \end{equation}
      and because we set $x^{\prime} = 0$, from Equation \ref{eq: lorentz_transfer_general}, 
      the two events of two successive photons collide with the electron can be 
      expressed as:
      $\left(\gamma v t_{1}^{\prime}, 0, 0, \gamma t_{1}^{\prime}\right)$ and 
      $\left(\gamma v t_{2}^{\prime}, 0, 0, \gamma t_{2}^{\prime}\right)$. 
      As Figure \ref{fig: inverse_compton_time_interval} shows, the time interval of two 
      successive photons (reciprocal of frequency) in frame $S$ is: 
      \begin{eqnarray}
        \label{eq: inverse_compton_time_interval}
        \Delta t &=& \left(t_2 - t_1\right) + \frac{\left(x_2 - x_1\right) \cos{\theta}}{c}  \nonumber \\ 
                &=& \gamma \left(t_{2}^{\prime} - t_{1}^{\prime}\right) + \frac{\gamma v \left(t_{2}^{\prime} - t_{1}^{\prime}\right) \cos{\theta}}{c} \nonumber \\
                &=&  \Delta t^{\prime} \gamma \left(1 + \beta \cos{\theta}\right) .
      \end{eqnarray}
      Equation \ref{eq: inverse_compton_time_interval} shows that the relationship of photon 
      number density between frame $S$ and $S^{\prime}$ is:
      \begin{equation}
          \label{eq: inverse_compton_number_density_relationship}
          n^{\prime} = n \gamma \left(1 + \beta \cos{\theta}\right) .
      \end{equation}
      Combine Equations \ref{eq: inverse_compton_number_density_relationship} and 
      \ref{eq: doppler_shift} we can transfer the incident photon energy density from 
      frame $S$ to $S^{\prime}$ according to Equation \ref{eq: inverse_compton_energy_density}
      \begin{equation}
        \label{eq: inverse_compton_energy_density}
        U_{rad}^{\prime} = U_{rad} \left[\gamma \left(1 + \beta \cos{\theta}\right)\right]^{2} .
      \end{equation}
      In Equation \ref{eq: inverse_compton_energy_density}, the incoming photon energy density 
      is a function of the incident angle ($\theta$), in order to get the total photon 
      energy density, we integrate the equation over $\theta$. Then we get: 
      \begin{equation}
        \label{eq: inverse_compton_energy_density_total}
        U_{rad}^{\prime} = \frac{4}{3} U_{rad} \left(\gamma^2 - \frac{1}{4}\right) .
      \end{equation}
      Combine Equations \ref{eq: inverse_compton_energy_density_total} and 
      \ref{eq: final_relationship}, the total scattered radiation power is:
      \begin{eqnarray}
        \label{eq: inverse_compton_power}
        P^{\prime} &=& P  \nonumber \\
                  &=& \frac{4}{3} \sigma_{T} c U_{rad} \left(\gamma^2 - \frac{1}{4}\right) .
      \end{eqnarray}
      As mentioned before, $P^{\prime}$ and $P$ are the total radiation powers after 
      scattering. Before the low energy photons gain energy, they give away some energy 
      first which is $\sigma_{T} c U_{rad}$. 
      Therefore, we have to subtract this value from Equation \ref{eq: inverse_compton_power} 
      to calculate the rate of net energy gain, which is described by Equation 
      \ref{eq: inverse_compton_net_gain}.
      \begin{eqnarray}
        \label{eq: inverse_compton_net_gain}
        P^{\prime} = P = \frac{dE}{dt} &=& \frac{4}{3} \sigma_{T} c U_{rad} \left(\gamma^2 - \frac{1}{4}\right) - \sigma_{T} c U_{rad} \nonumber \\
                                          &=& \frac{4}{3} \sigma_{T} c U_{rad} \beta^{2} \gamma^{2} .
      \end{eqnarray}
      If we compare Equation \ref{eq: inverse_compton_net_gain} with 
      Equation \ref{func: sync_combine}, we find that the form is very similar between these 
      two equations. 
      \begin{equation}
        \label{eq: comparision_inverse_compton_and_sync}
        \frac{P_{IC}}{P_{sync}} = \frac{U_{rad}}{U_{B}} ,
      \end{equation}
      where $U_{B}$ is the energy density of environment magnetic field. 

      \vspace{1cm}
      \singleFig{inverse_compton_time_interval}{0.45}{Diagram of Inverse Compoton Process.}
    
    \subsection{Curvature Radiation}
      Curvature radiation is the main source of gamma-ray photons. Charged particles  
      not only move around magnetic field lines (synchrotron radiation), 
      but also along magnetic field lines (curvature radiation) because the magnetic 
      field is very strong.

      Equation \ref{func: sync_total_power_precise} shows that total synchrotron radiation 
      power is related to pitch angle (because of the term $B_\perp$), and if the 
      pitch angle is $0$, there will be no synchrotron radiation. However, curvature 
      radiation can still be generated and is dependent on the curvature radii of 
      magnetic field lines:
      \begin{equation}
        \label{func: curvature_radiation}
        P = \frac{2}{3}\frac{e^2c}{s^2}\gamma^4 ,
      \end{equation}
      where $s$ is the curvature radii of magnetic field lines. 
      Equation \ref{func: curvature_radiation} is very similar to Equation 
      \ref{func: sync_total_power_simplified} and the only difference is that 
      $R$ is changed to $s$. Similarly, according to the Equation 
      \ref{func: syncrothron spectrum}, the power spectrum of curvature radiation can 
      be written as:
      \begin{equation}
        \label{func: curvature spectrum}
        P = \frac{\sqrt{3}e^2}{m_es}\gamma \left(\frac{\nu}{\nu_c}\right) \int_{\nu / \nu_c}^{\infty} K_{5/3}\left(\eta \right)d\eta ,
      \end{equation}
      where $\nu_c$ is the critical frequencies of curvature photons and equals to:
      \begin{equation}
        \label{func: curvature critical frequency}
        \nu_c = \frac{3}{4\pi}\frac{c}{s}\gamma^3 .
      \end{equation}
      According to Equations \ref{func: curvature_radiation}, \ref{func: curvature spectrum} 
      and \ref{func: curvature critical frequency}, the spectral properties of curvature 
      radiation are similar to synchrotron radiation. The only differences in the 
      equations are particles' Lorentz factors and curvature radii. 


  \section{Millisecond Pulsar} 
    \subsection{$\mbox{P}$-$\dot{P}$ Diagram} 
      $P$-$\dot{P}$ diagram is an important tool for analyzing evolutions of 
      pulsars. Period ($\mbox{P}$) and time derivative of period ($\dot{P}$) are 
      two of pulsars' important characteristics. Analyzing the position of a pulsar in 
      the $P$-$\dot{P}$ diagram can give us some valuable information such as 
      what evolution stage the pulsar is in or the type of the pulsar, etc. 
      Figure \ref{fig:p-pdot} is an example of a $P$-$\dot{P}$ diagram. 
      The horizontal axis is pulsars' rotation periods and the vertical axis is first time 
      derivative of rotation periods ($\dot{P}$).
      \begin{figure}[!htp]
          \centering
          \includegraphics[scale=0.3]{ppdot.png}
          \caption{\protect Positions of pulsars in $\mbox{P}$-$\dot{\mbox{P}}$ diagram.}
          \label{fig:p-pdot}
      \end{figure}
      In this $P$-$\dot{P}$ diagram, the negative slope lines represent the strengths 
      of surface magnetic fields while the positive slope lines represent the characteristic 
      ages of pulsars. The following is a short explanation for this. 
      From previous discussion, we have known that the 
      characteristic age of a pulsar is $\tau=-P/\dot{P}=P/(-\dot{P})$, so line of constant 
      $\tau$ is a set of straight lines with equal positive slope. Since 
      $B\propto\sqrt{P\dot{P}}$ the negative lines represent strengths of surface 
      magnetic fields.

      Figure \ref{fig:p-pdot} shows that most pulsars lie in the position of about 
      $1$s, $10^{-14}$s/s. At the same time, a couple of stars lie at the bottom-left 
      of Figure \ref{fig:p-pdot} --- these are millisecond pulsars (MSPs). Their 
      rotation periods are about 1-20 milliseconds. It is believed that MSPs are spun 
      up by accretion of mass from their companion stars. In the above 
      $P$-$\dot{P}$ diagram, we can observe that millisecond pulsars' 
      surface magnetic fields are about three to four orders of magnitude lower than those
      of normal pulsars. However, an MSP has a relative strong magnetic field near its 
      light cylinder ($B_{lc}$). For instance, PSR J1939+2134's $B_{lc}$ is about 
      $1.02\times10^5$ which is larger than Crab pulsar ($9.55\times10^5$). 
      \footnote{http://www.atnf.csiro.au/research/pulsar/psrcat/}
      % \cite{ATNF}
      The reason is that an MSP's radius 
      of light cylinder ($R_{lc}=c/\omega)$ is much smaller than a normal pulsar's 
      because of its short rotation period and the magnetic field near light cylinder can
      be estimated as $B_{lc}\sim\left(R/R_{lc}\right)^3$. At the same time, 
      pulsars' emission mechanism is closely related to their magnetic field near the light 
      cylinder. As a result, like a normal pulsar, an MSP also can have a broadband spectrum 
      from radio to gamma rays. 
    \subsection{Origin of Millisecond Pulsars}
      From pulsars' emission mechanisms, we know that the magnetic field of a pulsar 
      decreases with time while the spin period increases with time. But MSPs' spin 
      periods are much shorter than normal pulsars and surface magnetic fields are a 
      lot weaker. This makes an MSP seem to be both young and old. As a result,
      people think millisecond pulsars are old pulsars spun up by their companions. 
      The companion stars transfer mass and angular momentum to accelerate the rotation
      speed of pulsar. Therefore, the aged pulsar can spin faster gradually. 

      \subsubsection{Mass Transfer And Accretion In Binary Systems}
        X-ray binaries are a type of binary systems that is luminous in X-ray band. 
        There are several kinds of X-ray binaries including low mass X-ray binaries 
        (LMXB) and high mass X-ray binaries (HMXB). The mechanism of transferring mass 
        is different in these two types of systems. Before discussing mass 
        transfers, I will introduce Roche Lobe briefly. Figure 
        \ref{fig:roche lobe} is a schematic diagram of Roche Lobe.
        \begin{figure}[!htp]
          \centering
          \includegraphics[scale=0.5]{roche_lobes.jpg}
          \caption[A schematic diagram of Roche lobe.]
                  {A schematic diagram of Roche lobe. \protect $L_{1}$ is called inner 
                    Lagrange point which is the intersection of equipotential lines 
                    of the two stars.}
          \label{fig:roche lobe}
        \end{figure}
        We call two stars in an LMXB as A and B respectively for convenience. It is 
        obvious that if an object is close to star A, the gravitational influence of A 
        is so strong that the gravitational effect of the star B can be ignored. Similarly, 
        this is true for star B. As a result, there must be a point where the effect 
        of star A is equal to star B which is called inner Lagrange point.
        The two parts inside the largest equipotential lines of A and B are called Roche 
        lobe. If star B crosses its Roche lobe, then its mass will be attracted by A, 
        thus mass transfer between A and B happens. This is the main way of mass transfer 
        in LMXB. While in HMXB, the mass can be transferred by strong winds of the massive 
        companion star. 

        The process of mass transfer can change the distance between two 
        companion stars. If a low-mass star transfers mass to a high-mass companion star,
        the orbital separation will become larger. This can actually stop the mass transfer
        process and is a negative feedback. On the contrary, mass transfers from high-mass 
        star to low-mass star will shrink the orbital distance.
                       
    
  \section{Introduction to the Three MSPs and Previous Works}
    PSR B1937+21 is second fast spinning millisecond pulsar whose rotation period 
    is about $1.56$ms and was discovered in 1982 
    \cite{1982Natur.300..615B}, which makes it be the first discovered MSP. It has 
    extremely strong magnetic field at the light cylinder with about $10^6$ gauss, which 
    is larger than Crab pulsar. Its basic properties are listed in Table 
    \ref{table: basic_information_3_msps}. It has a double-peak pulse profile and the main 
    pulse is much stronger than the other \cite{j1939_pulse_profile}. It has very sharp 
    pulsations and nearly 100\% pulsed fractions. 
    
    Its gamma-ray properties were studied using \textit{Fermi} LAT and its 
    spectra were fitted by both a simple power-law model and a power-law with exponential
    cutoff (PLEC) model with photon-index of $2.38\pm0.07$ and $2.1\pm0.2$ respectively. 
    However, the researchers report that the simple power-law model is preferred according 
    to the convention in \cite{0067-0049-208-2-17} \cite{0004-637X-787-2-167}. In X-ray 
    energy band, PSR B1937+21 has been observed with \textit{XMM-Newton} and \textit{NuSTAR} 
    \cite{0004-637X-787-2-167,0004-637X-845-2-159}.
    
    PSR B1821$-$24 is in the globular cluster M28. There 
    are about a dozen pulsars discovered in M28, and PSR B1821$-$24 is the brightest one. 
    In addition, PSR B1821$-$24 is the most energetic MSP discovered whose spin-down
    luminosity is about $2.2\times 10^{36}$ erg $\mbox{s}^{-1}$. Like PSR B1937+21, it has 
    two very narrow pulses and the duty cycle is only about a few percent
    \cite{1538-4357-627-2-L125}.

    The gamma-ray spectrum of PSR B1821$-$24 is fitted by a PLEC model with a photon index of 
    $1.6\pm0.3$ using 44 months of \textit{Fermi} LAT data \cite{2013ApJ...778..106J}.
    \textit{NuSTAR} observed the globular cluster M28 in June 2015. Though \textit{NuSTAR}'s
    clock drift makes it hard to resolve the rotation period of PSR B1821$-$24, 
    researchers have invented a method the correct the photon arrival times
    \cite{0004-637X-845-2-159}.
    
    PSR J0218+4232 is also a very energetic MSP with a spin-down luminosity of $2.4\times
    10^{35}$ erg $\mbox{s}^{-1}$. Unlike the PSRs B1937+21 and B1821$-$24 which are solitary,
    it is accompanied by a white dwarf with a orbital period of two days at a distance of 
    about $3.15$ kpc \cite{article}. It was also studied using \textit{Fermi} LAT and 
    its spectrum is fitted by a PLEC model whose photon index is $2.0\pm0.1$ 
    \cite{0067-0049-208-2-17}.

    All three MSPs are very energetic and have very strong magnetic field at the light 
    cylinders as Table \ref{table: basic_information_3_msps} shows, which makes them can be 
    observed in radio, X-ray and gamma-ray bands like young pulsars. Additionally, they 
    show aligned pulse profiles in radio, X-ray and gamma-ray bands as Figure
    \ref{fig:class } shows. This may implies a different emission mechanism that radio,
    X-ray and gamma-ray emission is from the same location in a pulsar's magnetosphere. 
    The gamma-ray spectral properties of the three MSPs are listed in Table 
    \ref{table: previous_spectra_property}.  

    \begin{figure}[!htp]   
      \centering
      \includegraphics[width=6.9cm,height=7.6cm]{bands.png}
      \caption{Pulse profiles of PSR B1937+21 in radio, X-ray and gamma-ray bands. 
        The figure is adopted from \cite{0004-637X-787-2-167}}.
      \label{fig:class }
    \end{figure}
    
    \begin{table}[!htp]
      \centering
      \scalebox{0.8}{
      \begin{tabular}{|cccc|} 
        \hline 
        & J0218+4232 & B1821$-$24 & B1937+21 \\
        \hline 
        \hline 
        Distance (kpc) & $3.15$ & $5.50$ & $3.50$ \\
        Period ($\mbox{ms}$) & $2.32$  & $3.05$ & $1.56$ \\
        Period Derivative ($\dot{\text{P}}$, $10^{-20} $) & $7.74$  & $162$ & $10.5$ \\             
        Spin Down Age ($10^{8} $yr) & $4.76$ & $29.9$ & $2.35$ \\ 
        Surface Magnetic Field ($10^{8}$ G) & $4.29$ & $22.5$  & $4.09$ \\
        Light Cylinder Magnetic Field ($10^{5}$ G) & $3.21$ &$7.40$ & $10.2$ \\
        Spin Down Energy ($\dot{\text{E}}$, $10^{35}$ erg $\text{s}^{-1}$) & $2.4$ & $22$ & $11$\\ 
        \hline   
      \end{tabular}}
      \centering
      \caption[Spin and derived properties of PSRs J0218+4232, B1821$-$24 and B1937+21.]
        {Spin and derived properties of PSRs J0218+4232, B1821$-$24 and B1937+21. 
        The data are from the ATNF Pulsar Catalogue \protect \footnotemark}
      \label{table: basic_information_3_msps}
    \end{table}

    \footnotetext{http://www.atnf.csiro.au/research/pulsar/psrcat/}
    
    \begin{table}[!htp]
      \centering
      \scalebox{0.8}{
      \begin{tabular}{|cccc|} 
        \hline 
        & J0218+4232 & B1821$-$24 & B1937+21 \\
        \hline 
        \hline 
        Energy Range (GeV) & $0.1 \sim 100 $ & $0.1 \sim 100 $  & $0.5 \sim 6$ \\
        Photon Index $\Gamma$ & $2.0\pm0.1$ & $1.6\pm0.3$ & $1.43\pm0.87$   \\
        Cutoff Energy (GeV) & $4.6\pm1.2$ & $3.3\pm1.5$ & $1.15\pm0.74$\\
        Photon Flux ($10^{-8}$ $\text{cm}^{-2}$ $\text{s}^{-1}$)& $7.7\pm0.7$& $1.5\pm0.6$ & $1.22\pm0.23$ \\
        Energy Flux ($10^{-11}$ erg $\text{cm}^{-2}$ $\text{s}^{-1}$) & $4.56\pm0.24$ & $1.3\pm0.2$ & $1.98\pm0.32$ \\ 
        \hline   
      \end{tabular}}
      \vspace{0.5cm}
        \centering
        \caption[Spectra properties of the three MSPs from previous studies.]
          {Spectra properties of the three MSPs from previous studies
          \footnote{http://www.atnf.csiro.au/research/pulsar/psrcat/}
          \cite{0004-637X-787-2-167,J1939_old}.}
        \label{table: previous_spectra_property}
    \end{table}

  \section{Objectives}
    % In the paper (Guillemot \& Johnson et al. \cite{J1939_old}), they only study the PSR B1937+21 
    % with energy range from $0.1$ GeV to $6$ GeV and another paper (C.-Y. Ng \& Takata et al. 
    % \cite{0004-637X-787-2-167}) studied the MSP more 
    % thoroughly with a wider energy range. The paper proposed that a power-law model is 
    % preferred and the photon index is $2.38\pm0.07$ with a TS value of $112$. 
    
    Many gamma-ray studies about the three MSPs are a little bit old and now we have about 
    a lot more \textit{Fermi} LAT data. Besides, in 2015, Fermi team released Pass 8 data 
    including many improvements, such as better energy measurement and significantly 
    improved effective area. Additionally, there are new hard X-ray studies of the three 
    MSPs using \textit{NuSTAR} which provide better X-ray measurements.
    As a result, it is time to redo the gamma-ray analysis with the newer dataset and
    more gamma-ray data in order to gain more precise results. In addition, we also plan to 
    build a self-consistent emission model to explain the non-thermal high energy
    observations of the three MSPs. 

    Therefore, my main objective is to use the new data to redo the gamma-ray
    analysis of the three energetic MSPs mentioned above. Then, I will do a numerical
    simulation based on a theoretical model called two-layer model and test if the  
    predictions of the model are consistent with the new observational data. And finally, 
    based on numerical simulations of the two-layer emission model, I will generate 
    broadband spectra (including hard X-ray band and gamma-ray band) for all three MSPs. 

    
\chapter{Gamma-Ray Analysis and Results}
  As mentioned before, because of very short rotation periods, MSPs usually have very small 
  light cylinders compared with normal pulsars. Therefore, the magnetic field strength at 
  the light cylinders are comparable with young pulsars, especially for my target objects 
  --- PSRs J0218+4232, B1937+21 and B1821$-$24, which are among the fastest spinning and 
  most energetic MSPs. Thus, as normal pulsars, the three MSPs have broadband emissions so 
  it is intriguing to analyze the spectral properties of them in gamma-ray band.

  \section{Introduction to the \textit{Fermi Gamma-Ray} \\ \textit{Space Telescope}}
    The \textit{Fermi Gamma-Ray Space Telescope} was launched on June 11, 2008 and opened 
    a new window of studying supermassive black-hole systems, pulsars and so on. Its 
    original name was \textit{Gamma-Ray Large Area Space Telescope} (GLAST) and changed 
    to \textit{Fermi Gamma-Ray Space Telescope} in honor of the great scientist Enrico Fermi. 

    The \textit{Fermi Gamma-Ray Space Telescope} contains two parts: 
    Gamma-Ray Burst Monitor (GBM) and Large Area Telescope (LAT) and the 
    latter is the main instrument which is at least 30 times more sensitive than 
    all gamma-ray telescopes launched before. I only use LAT for my purposes. Thus, I 
    focus on the LAT instrument, which contains four main subcomponents including trackers, 
    calorimeters, anti-coincidence detectors and data acquisition systems. The reason why the 
    telescope is designed in this way is that high-energy gamma-rays cannot be refracted by
    lens or mirrors. Therefore, the working principles of the \textit{Fermi} LAT and other 
    gamma-ray telescopes are completely different.

    \begin{figure}[!htp]  
      \centering
      \includegraphics[scale=0.7]
              {Gamma_telescope_schematic.png}
      \caption{The figure (\blackhref{https://www-glast.stanford.edu/instrument.html}
              {https://www-glast.stanford.edu/instrument.html})
              illustrates how \textit{Fermi} LAT tracks incident gamma-ray photons.}
        \label{fig:fermi schematic}
    \end{figure}
    Figure \ref{fig:fermi schematic} demonstrates the very basic idea of the 
    \textit{Fermi} LAT working principles. 

    \begin{itemize}
      \item Gamma-ray photons can enter the anti-coincidence detector freely while 
        cosmic rays will generate signals which then tell the data acquisition system 
        component to reject these particles. In this way, the \textit{Fermi} LAT can 
        distinguish the gamma-ray photons from high energy cosmic rays with a confidence 
        level over 99.9\%.
      \item The conversion foil (Figure \ref{fig:fermi schematic}) can convert  
        gamma-ray photons into electron and positron pairs. This procedure makes it 
        possible to determine the directions of the incident gamma-ray photons. 
      \item The tracker (particle tracking detectors in Figure \ref{fig:fermi schematic}) 
        records the positions of the electrons and positrons generated from the gamma-ray 
        photons. There are many trackers so the paths of particles can be constructed by 
        numerical simulations.
      \item When the electrons and positrons reach the calorimeter, their energies can be
        measured. By calculating the energy relationships between the gamma-ray photons and 
        the corresponding electrons and positrons, the energies of the original gamma-ray 
        photons can also be obtained. 
      \item The data acquisition system is like a filter of gamma-ray photons which can 
        reject unwanted particles such as cosmic rays. Also, photons come from the Earth's 
        atmosphere are also rejected. 
    \end{itemize}

    For a telescope, the ability to measuring the directions and energies of incident 
    photons is very crucial. From the above descriptions of the \textit{Fermi} LAT 
    working principles, we know that the precision of construction of particles' paths 
    heavily influences how well we measure the directions and energies of gamma-ray 
    photons. And this process is greatly dependent on simulations and available Fermi datasets,
    which means that with the improvements of simulations and datasets,
    the precision and sensitivity of \textit{Fermi} LAT can also be improved. The Pass 8 data 
    have reprocessed the entire Fermi mission dataset, so the quality of the dataset is much 
    better. This is the main reason why I redo the analysis of the three MSPs.

  \section{A Brief Introduction to Fermi Data Analysis}
    \subsubsection{Processes of Doing Fermi Analysis}
      When doing \textit{Fermi} LAT data analysis, I basically dealt with two parts. The 
      first part is processing observational data and the second is generating photon 
      distributions based on spectra models. Cleaning data is straightforward including data 
      selection, data filtering with good time intervals (GTIs), generating count maps 
      and so on. Generating model-based count maps and count cubes needs a little bit 
      more efforts and mainly includes the following procedures. 

      Firstly, I need to generate a spectra model of all sources in our region of 
      interest (ROI). The model basically describes how strong each source is in 
      different energy bands and different positions. The initial parameters of the 
      model are from Fermi database. I do not fit positions of both point sources and 
      diffuse sources when doing the data analysis. However, the model alone is not very 
      helpful and I have to know other information in order to simulate photon 
      distributions discussed as the following.

      Since I am going to compare my simulation with the observational data, I have to take 
      the telescope status into account. For example, the effective area of the telescope 
      decreases when away from the optical axis. In addition, inclination angles and 
      observation time intervals have direct influences on the number of photon counts. 
      In short, after I get the simulated photon distribution from a model, it is 
      necessary to transfer the initial simulation into the real simulation by applying 
      the telescope functions. 
          
      After obtaining the photon distributions and spectra simulations, I then do 
      comparisons in order to get the maximum likelihood of the model. I can divide the 
      total energy band into many smaller energy bins and denote
      the number of photon counts in observational data as $n_{i}$, so that 
      $\sum_{i}^{}n_{i} = N$, where $N$ is the total number of photons in full energy range. 
      The observed number of photon counts in $ith$ bin is a Poisson distribution with a 
      mean value of $m_{i}$. In fact, the value $m_{i}$ is the expected number of photon counts 
      from our spectra model. Therefore, the distribution for $ith$
      bin can be expressed by Equation \ref{func: maximum_likelihood_poisson}, where 
      $P_{i}\left(n_{i}\right)$ is the possibility of observing the $n_{i}$ photon counts 
      for the $ith$ bin. 

      \begin{equation}
        P_{i}\left(n_{i}\right) = \frac{e^{-m_{i}} m_{i}^{n_{i}}}{n_{i}!} .
        \label{func: maximum_likelihood_poisson}
      \end{equation}

      As a result, it is not hard to generalize the possibility from each bin to all bins, 
      just by multiplying the possibilities for different bins.
      \begin{eqnarray}
        P_{total} &=& \prod_{i}^{}P_{i}\left(n_{i}\right)  \nonumber \\ 
                  &=& e^{-\sum_{i}^{}m_i}\prod_{i}^{}\frac{m_{i}^{n_i}}{n_i!} .
        \label{func: maximum_likelihood_poisson_all}
      \end{eqnarray}
      In Equation \ref{func: maximum_likelihood_poisson_all}, $n_i$ is directly 
      from observational data so they usually cannot be changed during the binned likelihood 
      analysis. However, by changing the model, the $m_i$ can be altered. 
      Hence, my aim is to tweak the spectra model in order to make the total possibility 
      $P_{total}$ as large as possible. 

      This is the basic idea and procedure of doing Fermi data analysis. After doing these,
      I go further such as testing how significant the targets are by creating TS 
      maps. The thesis basically follows the procedures. 

      Before finishing this part, I should briefly introduce the basic idea of TS maps. 
      TS value stands for Test Statistic value which can be expressed as Equation 
      \ref{func: ts_definition} 
      \begin{equation}
        TS = -2 \ln{\frac{L_{max,0}}{L_{max,1}}} ,
        \label{func: ts_definition}
      \end{equation}
      where $L_{max,0}$ and $L_{max,1}$ are the maximum likelihood of models in which the 
      target source is not included and included respectively. According to Equation 
      \ref{func: ts_definition}, the larger the TS value is, the larger $L_{max, 1}$ is, 
      which means that the probability of existence of the target source is larger. 
      In order to generate a TS map, I divide the whole map into many sub-grids. In each 
      sub-grid, the ``gtlike'' algorithm basically does two things. The first procedure is 
      calculating the maximum likelihood value directly based on the spectra model 
      ($L_{max,0}$). Then it adds an imaginary point source in the sub-grid, fits the source 
      and gets the maximum likelihood ($L_{max, 1}$) value. Therefore, I get two 
      maximum likelihood values. In the end, it calculate the TS value for the 
      sub-grid according to Equation \ref{func: ts_definition}. 

      After calculating the TS values for all sub-grids, I generate a TS map just by 
      rendering colors according to each grid's TS value. By comparing TS values of 
      all sub-grids in a TS map, I can determine where the target source is most likely to 
      be and how large the probability is. 
      
      Generally speaking, for each source, I generate two TS maps with and without the 
      target source respectively. Then I determine how likely my target source is observed 
      by calculating the TS values and compare the two TS maps. 
      For instance, if the data show the source is observed, 
      then the value of each pixel of the TS map containing the source should be low because 
      the probability of adding an imaginary point source is low. 
      On the contrary, the TS values of the pixels around the position of the target 
      source should be significantly higher than other positions in the TS map if the target 
      source is deleted from the fitted spectra model.
       
  \section{\textit{Fermi} LAT Data Analysis}
    The basic idea of fitting spectral parameters is to make the count cube generated by 
    the model be as similar to the observational data as possible. A count cube is just a 
    combination of many count maps in different energy bands. For example, a dataset whose 
    energy range is from $100\mbox{MeV}$ to $100\mbox{GeV}$ can be divided into 30 bins. 
    I generate a count map in each energy bin, thus I have 30 count maps.
    
    A count map is basically generated by the following steps. Firstly, choose a 
    pixel of a certain size. Then check each photon's direction to determine if 
    the photon is in this pixel. If it is in the pixel, the photon counts of the 
    pixel will add one. Therefore, the more photons fall within the pixel, the more 
    photon counts the pixel has, hence the brighter the pixel is. By doing the same 
    thing for every pixel in the ROI, a count map is generated. A count map shows 
    what has been observed intuitively and offers a very basic idea of if 
    the desired data is processed rightly.
          
    The calculation process can be summarized as follows. First of all, I have to generate 
    a spectra model for every known source in the region of interest (ROI) based on the Fermi 
    database. The database includes \textit{LAT} four-year Point Source Catalog (3FGL), 
    Galactic diffuse emission (gll\_iem\_v06.fits) and the isotropic emission 
    (iso\_P8R2\_SOURCE\_V6\_v06.txt). Then I produce a count cube based on the 
    model. Generally speaking, the differences of the count cubes between the model and 
    observation is obvious. Then, the Fermi software adjusts the parameters 
    to make the difference smaller. Until the errors are acceptable, the software 
    outputs the final fitted parameters of corresponding spectra models.    

    I use a power-law with exponential-cutoff (PLEC) model to fit the 
    observational data and it is a special case of the power-law with 
    super-exponential-cutoff (PLSuperExpCutoff) model. The spectra of PLSuperExpCutoff 
    can be described by Equation \ref{eq: fermi_model}:  
    \begin{equation} 
      \label{eq: fermi_model}
      \frac{dN}{dE} = N_{0} \left(\frac{E}{E_0}\right)^{\gamma_1}\mbox{exp}\left[-\left(\frac{E}{E_c}\right)^{\gamma_2}\right] ,
    \end{equation}  
    where $N_0$ is called prefactor, $E_c$ is the cutoff energy and the $E_0$ is a scale 
    parameter. PLEC model is the special case where $\gamma_2=1$. My aim is to 
    fit the parameters $N_0$, $E_c$ and $\gamma_1$ to make the model be more 
    consistent with the \textit{Fermi} LAT observational data.

    \subsection{Verifying the Data Analysis Process}
      Before analyzing the observations of my target sources, it is reasonable to test 
      if my procedures for data processing are right. In order to do so, I try to do 
      analysis for two bright pulsars PSRs J0007+7303 and J0534+2200. The reason I choose 
      these two pulsars is that according to previous studies, they are bright and easy 
      to detect with a large TS value of 43388 and 102653 for J0007+7303 and J0534+2200 
      respectively \cite{0067-0049-208-2-17}.

      In the spectra fit process, I do not use the same fit parameters as the previous 
      paper (for instance, the number of free parameters in the ROI is different), 
      however, I get similar results in terms of spectral index.  
      In Table \ref{table: previous_result_comparison}, I used the observational data from 
      2009-01-01 to 2013-02-01 in order to try to be consistent with the old results 
      \cite{0067-0049-208-2-17}. In addition, I also fit spectra with 
      observational data up to 2018-02-01 and Pass 8 dataset to test how big improvement 
      I can make with the new Fermi Pass 8 dataset and more observational data. 
      The results of year 2018 data are shown in Table \ref{table: 2018_fit_data}.

      Tables \ref{table: previous_result_comparison} and \ref{table: 2018_fit_data} mainly 
      show two pieces of information. Firstly, my procedures for dealing with observation 
      data have no obvious problems, so basically I can trust fit results of my target 
      sources. Secondly, the \textit{Fermi} Pass 8 LAT Data has improved the accuracy a 
      lot. For example, as Table \ref{table: previous_result_comparison} shows, the photon 
      indexes are $1.30\pm0.02$ and $1.4\pm0.1$, which shows that the errors reduce a lot. 
      Additionally, the TS value is more than double as before.
      % \question{However, the cutoff energies are not consistent between the previous result and the new result.
        % (I need to explain this a little bit later).}
      \vspace{1cm} 
      \begin{table}[!htp]
        \centering
        \scalebox{0.8}{
        \begin{tabular}{|c|c|c|c|c|c|c|} 
          \hline 
          & \multicolumn{3}{|c|}{Test Results} & \multicolumn{3}{|c|}{Previous Results} \\ 
          \cline{2-7}
          & $\Gamma$ & $E_c$ (MeV) & TS & $\Gamma$ & $E_c$ (MeV) & TS \\ 
          \hline
          J0007+7303 & $1.30\pm0.02$ & $2010\pm85$ & $96979$ & $1.4\pm0.1$ & $4700\pm200$ & $43388$  \\
          \hline 
          J0534+2200 & $2.07\pm0.01$ & $9880\pm572$ & $239015$ & $1.9\pm0.1$ & $4200\pm200$ & $102653$  \\
          \hline
        \end{tabular}}
        \vspace{0.5cm}
          \centering
          \caption[The spectra fit results with data from 2008 August 4 to 2011 August 4.]
            {The spectra fit results with data from 2008 August 4 to 2011 August 4. 
            In the thesis, in order to make data 
            analysis more convenient, I use some pipeline scripts to deal with the 
            observational data. The "Test Results" column is the results generated by 
            using the pipeline scripts. The "Previous Results" column lists the 
            corresponding spectral properties based on the previous studies 
            \cite{0067-0049-208-2-17}. According to the standard 
            PLEC model (described in equation \ref{eq: fermi_model}), $\Gamma$ is 
            photon index and $E_c$ is cutoff energy.}
          \label{table: previous_result_comparison}
      \end{table}
      \vspace{1cm}            

      \begin{table}[!ht]
        \centering
        \scalebox{0.85}{
        \begin{tabular}{|cccc|}
          \hline 
          &$\Gamma$& $E_c$ (GeV) & TS Value\\ \hline \hline
          J0007+7303 & $1.34\pm0.02$ & $2.20\pm0.67$ & $210166$ \\  
          J0534+2200 & $2.01\pm0.01$ & $9.20\pm0.37$ & $449946$ \\
          \hline
        \end{tabular}}
        \caption[Fit results with data from year 2009 to year 2018.]
          {Fit results with data from year 2009 to year 2018. The physical 
          meanings of $\Gamma$ and $E_c$ are the same as Table 
          \ref{table: previous_result_comparison}}
        \label{table: 2018_fit_data}
      \end{table}
      % \vspace{1cm}            

  \section{PSR J0218+4232}
  
    The region of interest (ROI) is a circle with a radius of $20^\circ$ and all parameters of
    sources which are $8^\circ$ outside of the center are fixed. For sources within 
    $8^\circ$, initial values of parameters are the same as their default values according 
    to \fgl. In this case, there are seven point sources which have free parameters. In 
    Figure \ref{fig: j0218_count_map_and_model}, the green circles represent those sources.
    There are some very bright sources which have no free parameters
    in the outer parts of the count map. The reason is that they are so far away 
    from PSR J0218+4232 that \textit{Fermi} LAT can distinguish if a photon comes 
    from the target source or other outer sources. As a consequence, I do not need to fit 
    any parameters for those outer sources and their spectral properties are from 
    \fgl{}. However, it is a different case for the nearby sources and they have to be fitted
    with the model.
          
    \subsection{Count Maps And Count Cubes}
      \begin{figure}[!ht]  
        \begin{center}
        \begin{minipage}{0.45\textwidth}
          \begin{center} 
              \includegraphics[scale=0.33]{j0218_count_map_with_region.png}
          \end{center}
        \end{minipage}
        \begin{minipage}{0.45\textwidth}
          \begin{center} 
              \includegraphics[scale=0.33]
                  {j0218_count_map_model.png}
          \end{center}
        \end{minipage}
      \end{center}
      \begin{center}
        \caption[The count maps of observational data and the spectral model.]
        {The count map of PSR J0218+4232 (left) and the count map generated by 
        the model (right). The target source is in black circle. In the left panel, 
        the green circles represent sources needed 
        to be fitted. The right panel is a count map created according to the fitted 
        spectra model. The size of each figure is 141 pixels $\times$ 141 pixels, 
        and the dimension for each pixel is $0.2^\circ \times 0.2^\circ$.}
        \label{fig: j0218_count_map_and_model}  
        \end{center} 
      \end{figure}

      The left panel of Figure \ref{fig: j0218_count_map_and_model} is the count map of 
      PSR J0218+4232 created directly from the observational data. In the center of the left 
      panel, the target source can be seen clearly.
      The dimensions of the figures seem to be weird and the reason why the count 
      map is $141$ pixels wide is that I need to select a circle region from the original 
      data. However, when generating a count map, I have to assign the sizes for x and y 
      axis separately, which means that a count map is actually rectangular. As a 
      consequence, I have to crop a rectangular from the original circle region and usually, 
      the rectangular is chosen as a square. 
    
      \begin{figure}[!htp]
        \begin{minipage}{0.32\textwidth}
          \begin{center} 
            \includegraphics[scale=0.28]
                  {j0218_ccube_start.png}
          \end{center}
        \end{minipage}
        \begin{minipage}{0.32\textwidth}
          \begin{center}
            \includegraphics[scale=0.28]
                  {j0218_ccube_middle.png}
          \end{center}
        \end{minipage}
        \begin{minipage}{0.32\textwidth}
          \begin{center}
          \includegraphics[scale=0.28]
                {j0218_ccube_end.png}
          \end{center}
        \end{minipage}
        \caption[Three count maps from PSR J0218+4232's count cube.]
          {Three count maps from PSR J0218+4232's count cube. The energy ranges of 
          the figures are 100$\sim$123MeV (left), 1.873$\sim$2.310GeV (middle), 
          35.11$\sim$43.29GeV (right).}
        \label{fig: j0218_ccube_bin_1_and_15}
      \end{figure}
            
      Figure \ref{fig: j0218_ccube_bin_1_and_15} is a comparison between PSR J0218+4232's 
      count maps in different energy bands. The count map of about $100\mbox{MeV}$ is too  
      noisy to distinguish the target source while the energy is above 
      $30\mbox{GeV}$ there are so few photons that there is not a clear sign of the source. 
      I choose three circle regions whose centers are the target sources and the radii 
      are 1000 $''$ for all three figures and then calculate the total numbers of 
      photon counts of the selected regions. 
      As Table \ref{table:j0218_ccube_photon_counts} shows, though the total number of photon 
      counts around the target source is similar between the left and middle count maps, 
      the numbers of counts per energy are much different. Since there are few photons 
      in high energy bands (above $50\mbox{GeV}$) compare to other energy bands, I focus 
      more on the lower energy part. 

      \begin{table}[!htp]   
        \centering
        \scalebox{0.85}{
        \begin{tabular}{|m{4.5cm}ccc|}
          \hline 
          Energy range (MeV) & 100$\sim$123 & 1873$\sim$2310 & 35110$\sim$43290 \\
          \hline \hline 
          Total counts & 78 & 93 & 0 \\
          Counts / MeV ($\text{MeV}^{-1}$)& 3.39 & 0.21 & 0.00 \\  
          \hline
        \end{tabular}}
        \mycaption{Numbers of photon counts of count maps in different energy bands for 
              PSR J0218+4232.}
        \label{table:j0218_ccube_photon_counts}
      \end{table}

    \subsection{Binned Likelihood Analysis}
      Figure \ref{fig: j0218_count_map_and_model} shows that the fit results of the model 
      are consistent with the observation. However, there are lots of small red pixels 
      in the left panel (generated directly by the observational data) while the 
      right panel is very "clean". This means that a lot of photons are thought as 
      generated by the modeled source. Thus, in the model, the sources are generally 
      slightly brighter than the observation. However, the target
      source is an exception. In the region I have used before (the center is the 
      target source, and the radius is 1000$''$), the total photon counts in the left 
      panel are 1815 compare to 1737 in the right panel. 

      The reason why the count map generated directly by the observational data is a lot 
      more messy is that the source model is generated based on the Fermi database 
      and all sources' spatial positions are fixed. This means that if 
      a photon comes from a particular direction and there is no any known pulsar in 
      that direction, this photon has to be classified into other directions and there 
      is a modeled source in the direction.  Thus, the spatial positions of photons are 
      different between the observation and the model, and the count maps generated 
      directly from models are usually cleaner. 
      \begin{figure}[!htp]
        \begin{center}
        \begin{minipage}{0.45\textwidth}
          \begin{center} 
            \includegraphics[scale=0.4]
                  {j0218_count_map_linear_scale.png}
          \end{center}
        \end{minipage}
        \begin{minipage}{0.45\textwidth}
          \begin{center}
            \includegraphics[scale=0.34]
                  {j0218_dif_map_linear_scale_new.png}
          \end{center}
        \end{minipage}
      \end{center}
      \caption[The count map and residual map of PSR J0218+4232.]
        {The count map and residual map of PSR J0218+4232.
        The figures are both in linear scale in order to compare the residual map 
        with the original count map more intuitively. The left panel is the count 
        map and the right panel is the residual map which shows the differences between 
        the observation and the spectra model. It is created by directly subtracting 
        the photon counts of each pixel between the count maps of observational data and 
        the spectra model. The green circle region represents (the regions are 
        completely the same in the two figures) the largest number photon counts of the 
        residual map and its radius is $2000''$.}
      \label{fig: j0218_count_map_diff}
      \end{figure}

      Figure \ref{fig: j0218_count_map_diff} basically describes how well the model is 
      compared to the observational data. There are some bright dots in 
      the residual map showing the differences between the spectra model and the 
      observational data. In the residual map of Figure \ref{fig: j0218_count_map_diff}, 
      most differences of absolute photon counts are small, however, in the green 
      circle region, the absolute value 6003 is large. This means that in this region,
      the number of photon counts of the observational data (21525) is 6003 larger than 
      in our model. This is not negligible since it is nearly $28\%$ of the original 
      photon counts. Does this mean that the model is not good? The answer should be 
      yes, however, this does not mean the fit is not good since the model parameters 
      in this region are all fixed and the fixed values are from the \textit{Fermi}
      LAT four-year Point Source Catalog. Hence, the difference shows some problems of the 
      spectra model, but has nothing to do with the fit results. Instead, from the 
      residual map, we see that the fit results are good because the differences
      of the number of photon counts are very small, which are about $5\%$ of the photon 
      counts of the count map on average.

      Table \ref{table: j0218_fit_result} lists the results of the fit parameters. 
      We see from Table \ref{table: j0218_fit_result} that the new fit results are 
      consistent with the old ones. However, the precision improves a lot which is 
      ascribed to the \fgl{} and PASS 8 dataset. Figure \ref{fig: j0218_cur.png} is a 
      plot of the spectrum according to Function \ref{eq: fermi_model}.
      One thing should be noticed is that I need to multiply $E^2$ to Function 
      \ref{eq: fermi_model} 
      in order to get the flux. Figure \ref{fig: j0218_cur.png} shows that 
      the global fit is consistent with flux points fitted by each energy bin separately. 
      The TS value of PSR J0218+4232 is 7110, which gives 
      a significance level $\sigma \approx \sqrt{TS} \approx 84$. This strongly implies the 
      presence of the target source. I also plot TS maps to test the presence of the 
      source as Figure \ref{fig: j0218_tsmap_comparison_20} shows. 
            
      \begin{table}[!htp]
        \centering
          \scalebox{0.8}{
          \begin{tabular}{|ccc|}
            \hline
            & This Study & Previous Results \\
            \hline \hline 
            Photon Index ($\Gamma$) & $1.89\pm0.04$ & $2.0\pm0.1$ \\
            Cutoff ($E_c$, GeV) & $3.77\pm0.40$ & $4.6\pm1.2$ \\
            Photon Flux ($10^{-8}$ $\text{cm}^{-2} \text{s}^{-1}$) & $7.29\pm0.28$ & $7.7\pm0.7$ \\
            Energy Flux ($10^{-11}$ erg $\text{cm}^{-2} \text{s}^{-1}$) & $4.45\pm0.16$ & $4.56\pm0.24$ \\
            TS value & $6809$ &  $1313$  \\
            \hline
          \end{tabular}}  
          \caption[Fit parameters of the spectra model of PSR J0218+4232.]
            {Fit parameters of the spectra model of PSR J0218+4232. 
            The names of parameters are consistent with Equation
            \ref{eq: fermi_model}. The previous results are reported by
            \cite{0067-0049-208-2-17}}
          \label{table: j0218_fit_result}        
      \end{table}  

      \begin{figure}[!htp]
        \centering 
        \includegraphics[scale=0.35]{j0218_cur.png}
        \caption[The log-log plot of flux to energy of PSR J0218+4232.]
        {The log-log plot of flux to energy of PSR J0218+4232. The grey shade represents 
        fit errors, black points with error bars are flux points, the blue dots are upper values and the 
        red line is the PLEC model multiplied by $E^2$. Flux points 
        are fitted separately by dividing the total energy bin (100 MeV $\sim$ 100 GeV) into multiple energy bins.
        The horizontal error bars represents the width of each bin.}
        \label{fig: j0218_cur.png}
      \end{figure}
        
      % \singleFig{j0218_cur.png}{0.35}{The log-log plot of flux to energy of PSR J0218+4232. The grey shade represents 
      %   fit errors, black points with error bars are flux points, the blue dots are upper values and the 
      %   red line is the PLEC model multiplied by $E^2$. Flux points 
      %   are fitted separately by dividing the total energy bin (100 MeV $\sim$ 100 GeV) into multiple energy bins.
      %   The horizontal error bars represents the width of each bin. }
      \vspace{1cm}
      \begin{figure}[!htp]
        \begin{center}
        \begin{minipage}{0.46\textwidth}
          \begin{center} 
            \includegraphics[scale=0.37]{j0218_tsmap_with_source_20.png}
          \end{center}
        \end{minipage}
        \begin{minipage}{0.45\textwidth}
          \begin{center}
            \includegraphics[scale=0.37]{j0218_nosource_20.png}
          \end{center}
        \end{minipage}
      \end{center}
      \caption[TS maps of PSR J0218+4232.]
        {TS maps of PSR J0218+4232. The figures' dimensions are 
        $4^{\circ} \times 4^{\circ}$ ($20$ pixels $\times$ $20$ pixels with 
        $0.2^{\circ} \times 0.2^{\circ}$ for each pixel). The left and the right panels are 
        generated by the xml models with and without the target source PSR J0218+4232 respectively.
        The left panel shows that the possibility of adding an imputative point source is very low 
        only with a maximum TS value of less than 5. However, the right panel strongly implies that 
        there should be an additional source after I have removed the target MSP from the spectra 
        model, which means it's highly likely that PSR J0218+4232 is contained in the observation 
        data.}
      \label{fig: j0218_tsmap_comparison_20}
      \end{figure}

      It is also instructive to check the count residuals of the fit as Figure 
      \ref{fig: j0218_count_spectra} shows. The galactic and isotropic emissions are very 
      bright compared with the target source and this can bring some difficulties to 
      the fit (especially for the other two MSPs PSRs B1937+21 and B1821$-$24). The residuals 
      and fluctuations become larger and more obvious when the energy is larger than 
      $10\mbox{GeV}$. At the same time, in low gamma-ray part 
      (from $100\mbox{MeV}$ to $200\mbox{MeV}$), the number of counts of the model 
      deviates from observation counts obviously because of the relatively low energy 
      resolution in the low-energy gamma-ray part. The error of the first bin is larger
      than the next several bins in Figure \ref{fig: j0218_cur.png} also shows that 
      the fit in the low-energy gamma-ray band (from $100\mbox{MeV}$ to $200\mbox{MeV}$) 
      is not as good as higher energy bands (but not too high). 
            
      \begin{figure}[!htp]
        \centering
        \includegraphics[scale=0.42]{j0218_count_spectra.png}
        \caption[The count spectra and count residuals of PSR J0218+4232.]
                {The count spectra and count residuals of PSR J0218+4232.
                The upper panel is the count spectra of all sources included in the 
                fit procedure. Thick lines are those sources with free fit parameters 
                while the thin lines are fixed sources. The green line and the black 
                dots represent observed counts in different energy bands. The purple 
                line represents galactic emissions. The lower panel shows the count 
                residuals in different energy bands. } 
        \label{fig: j0218_count_spectra}
      \end{figure}

  \section{PSR B1821$-$24}
    The ROI region is also a circle whose radius is $20^\circ$ and all 
    parameters of sources outside of $8^\circ$ are fixed. 
    There are six free sources in the region of $8^\circ$. Figure
    \ref{fig: b1821_count_map_with_region_and_model} 
    is a combination of count maps of observational data and the model. 

    \subsection{Count Maps And Count Cubes}
      \begin{figure}[!ht]
        \begin{center}
        \begin{minipage}{0.45\textwidth}
          \centering 
          \includegraphics[scale=0.27]{b1821_count_map_with_region.png}
        \end{minipage}
        \begin{minipage}{0.45\textwidth}
          \centering
          \includegraphics[scale=0.27]{b1821_count_map_model.png}
        \end{minipage}
      \end{center}
      \caption[The count map of PSR B1821$-$24 (left) and the count map generated by 
              the model (right).]
        {The count map of PSR B1821$-$24 (left) and the count map generated by 
        the model (right). The green circles are sources with free parameters and the black circle
        is PSR B1821$-$24. The sizes of the both figures are 141 pixels $\times$ 141 pixels, and 
        each pixel's dimension is $0.2^\circ \times 0.2^\circ$.}
      \label{fig: b1821_count_map_with_region_and_model}
      \end{figure}
  
      \begin{figure}[!ht]
        \begin{center}
          \begin{minipage}{0.31\textwidth}
            \begin{center} 
              \includegraphics[scale=0.27]{b1821_ccube_start.png}
            \end{center}
          \end{minipage}
          \begin{minipage}{0.31\textwidth}
            \begin{center}
              \includegraphics[scale=0.27]{b1821_ccube_middle.png}
            \end{center}
          \end{minipage}
          \begin{minipage}{0.31\textwidth}
            \begin{center}
            \includegraphics[scale=0.27]{b1821_ccube_end.png}
            \end{center}
          \end{minipage}
          \end{center}
          \caption[Three count maps of PSR B1821$-$24's count cube.]
            {Three count maps of PSR B1821$-$24's count cube. The energy ranges of the 
            figures are 100$\sim$123MeV, 1.873$\sim$2.310GeV, 81.11$\sim$100GeV respectively from 
            left to right.}
          \label{fig: b1821_ccube_1_15_33.png}
      \end{figure}

      The left and right panels of Figure \ref{fig: b1821_count_map_diff.png} are the count 
      map of the PSR B1821$-$24 generated from observational data and spectra model respectively. 
      Like the situations of PSR J0218+4232, the count map from the model is clearly cleaner than 
      from the observational data and the two figures are very similar, which implies that 
      the spectra model describes the observational data well.  

      Figure \ref{fig: b1821_ccube_1_15_33.png} are count maps of PSR B1821$-$24 in 
      different energy bands. The target pulsar is too faint in very high energy bands 
      and interfered too much by the ambient environment in low energy bands 
      (slightly above 100MeV). PSR B1821$-$24 is in the M28 globular cluster and is the most 
      energetic one, which is much brighter than other sources found in M28. However, since 
      PSR B1821$-$24 is very faint observed from Earth, it is understandable that the fit 
      results are not as good as the results of PSR J0218+4232. 
    \subsection{Binned Likelihood Analysis}
      \begin{figure}[!ht]
        \begin{center}
        \begin{minipage}{0.45\textwidth}
          \begin{center} 
            \includegraphics[scale=0.40]{b1821_countmap_noregion.png}
          \end{center}
        \end{minipage}
        \begin{minipage}{0.45\textwidth}
          \begin{center}
            \includegraphics[scale=0.335]{b1821_count_map_diff_new.png}
          \end{center}
        \end{minipage}
      \end{center}
      \caption[The count map and residual map of PSR B1821$-$24 in linear scale.]
        {The count map and residual map of PSR B1821$-$24 in linear scale. 
        The left panel is the count map and the right panel is the residual map showing the 
        difference between the observational data and the spectra model.}
        \label{fig: b1821_count_map_diff.png}
      \end{figure}

      \singleFig{b1821_cur.png}{0.37}{The log-log plot of flux to energy of 
        PSR B1821$-$24's gamma-ray spectrum. }
      \vspace{1cm}

      The differences of the count map between the observational data and the model are described 
      as Figure \ref{fig: b1821_count_map_diff.png} which is in linear scale. Although there are 
      many red and blue dots in the right panel of the figure, their absolute values are 
      generally small compared with the original counts value.
      Thus, the fits are acceptable in general. But I am still trying to get the better spectra 
      model. 

      Table \ref{table: b1821_fit_result} shows the global fit results of PSR B1821$-$24. 
      The TS value of the model is 941 which gives us a significance level 
      of about $\sqrt{941} \sim 31$. This strongly supports the existence of the target source in 
      the observational data. As Table \ref{table: b1821_fit_result} shows, the energy flux from 
      100MeV to 100GeV is not consistent between the two studies and gamma-ray spectrum of the 
      previous study is also softer \cite{2013ApJ...778..106J}.

      Figure \ref{fig: b1821_twolayer_cur.png} shows that the global fit is consistent with 
      the flux points generated by fitting sub-energy bins separately. We notice the upper value 
      for the first energy bin is slightly smaller than the global fit. 
      Though it is strange that the upper value is smaller than the normal value at first glance, 
      it is reasonable since the flux points are fitted separately and are independent to 
      the global fit. In fact, I use a single power-law model to fit each sub-energy bin while 
      PLEC model to do the global fit. As discussed previously, the lower energy parts of 
      the observation (slightly above 100MeV) is not as reliable as other energy bands. As a result, the 
      separate fit for the first energy bin is not as good as the global fit and it is reasonable 
      that the two fit results are not completely consistent. When this happening, I have 
      more confidence in the global fit than the separate fit.
            
      Figure \ref{fig: b1821_tsmap_comparison_20} contains TS maps of PSR B1821$-$24. The 
      comparison of TS maps also show the significant of the target MSP. Figure 
      \ref{fig: b1821_count_spectra} shows how well the fit is. Like the count residuals of 
      the other two pulsars, the fit is not good in low energy part (from 100MeV to 500MeV)
      and high energy band (above 10GeV). Besides, as Figures \ref{fig: j0218_count_spectra},
      \ref{fig: b1821_count_spectra} and \ref{fig: j1939_count_spectra} show, the numbers of 
      observed photon events are all larger than modeled photon counts for PSRs J0218+4232, 
      B1821$-$24 and B1937+21. 

      \begin{table}[!htp]
        \centering
          \scalebox{0.8}{
          \begin{tabular}{|ccc|}
            \hline 
            & This Study & Previous Results \\
            \hline \hline  
            Photon Index ($\Gamma$) & $1.91\pm0.07$ & $1.6\pm0.3$ \\
            Cutoff Energy($E_c$, GeV) & $4.50\pm0.71$ & $3.3\pm1.5$ \\
            Photon Flux ($10^{-8}$ $\text{cm}^{-2} \text{s}^{-1}$) & $3.85\pm0.31$ & $1.5\pm0.6$ \\ 
            Energy Flux ($10^{-11}$ erg $\text{cm}^{-2}$ $\text{s}^{-1}$) &$2.44\pm0.14$ & $1.3\pm0.2$\\
            TS value & $941$ & $76$ \\
            \hline
          \end{tabular}}  
          \caption[Fit parameters of the spectra model of PSR B1821$-$24.]
            {Fit parameters of the spectra model of PSR B1821$-$24. 
            The names of parameters are also consistent with Equation
            \ref{eq: fermi_model}.}
            
          \label{table: b1821_fit_result}        
      \end{table}  
      \vspace{1cm}
          
      \begin{figure}[!ht]
        \begin{center}
        \begin{minipage}{0.46\textwidth}
          \begin{center} 
            \includegraphics[scale=0.35]{b1821_tsmap_with_source_20.png}
          \end{center}
        \end{minipage}
        \begin{minipage}{0.45\textwidth}
          \begin{center}
            \includegraphics[scale=0.35]{b1821_tsmap_nosource_20.png}
          \end{center}
        \end{minipage}
      \end{center}
        \caption[TS maps of PSR B1821$-$24.]
        {TS maps of PSR B1821$-$24. The figures' dimensions are $4^{\circ} \times4^{\circ}$ 
        ($20$ pixels $\times$ $20$ pixels with $0.2^{\circ} \times 0.2^{\circ}$ for each pixel). 
        The left and right panels are generated by the XML models with and without the 
        target source PSR B1821$-$24 respectively. The left panel shows that the possibility 
        of adding an imputative point source is very low only with a maximum TS value of 
        less than 11 while the TS values of the right figure are generally much larger.}
          \label{fig: b1821_tsmap_comparison_20}
      \end{figure}
      \vspace{1cm}
            
      \begin{figure}[!htp]
        \centering
        \includegraphics[scale=0.40]{b1821_count_spectra.png}
        \caption[The count spectra and count residuals of PSR B1821$-$24.]
                {The count spectra and count residuals of PSR B1821$-$24.
                The upper panel is the count spectra of all sources included in the 
                ROI. Thick lines are those sources with free fit parameters 
                while the thin lines are fixed sources.} 
        \label{fig: b1821_count_spectra}
      \end{figure}


  \section{PSR B1937+21}
    \subsection{Phase Averaged Analysis}
      First of all, I use a PLEC model to fit the gamma-ray spectra. 
      In order to make the data analysis be more consistent, I choose the same parameters
      to process the raw observational data. Like the other two MSPs, the radius of the ROI 
      is $20^{\circ}$ degrees, and all parameters of sources $8^{\circ}$ degrees outside 
      from the center are fixed with default values. There are nine point sources including 
      the target source PSR B1937+21 and twenty-eight free parameters. In fact, PSR B1937+21
      is not included in \fgl{} may be because that the MSP is very weak and the signal to 
      noise ratio is so low that the reliable spectrum fit results with a large TS value 
      had not been obtained when the \fgl{} was published. Thus I have to add the 
      configuration file for the MSP manually and I set the initial value of photon index 
      to be $-2.0$.

      \subsubsection{Count Maps and Count Cubes}
        Figure \ref{fig: j1939_count_map_ave} is the comparison of count maps between 
        observational data. Like the previous conditions, the count map of observational data 
        is more messy than the fitted model. 

        \begin{figure}[!ht]
          \begin{center}
          \begin{minipage}{0.45\textwidth}
            \begin{center} 
              \includegraphics[scale=0.33]{j1939_cmap_ave.png}
            \end{center}
          \end{minipage}
          \begin{minipage}{0.45\textwidth}
            \begin{center}
              \includegraphics[scale=0.33]{j1939_model_map_ave.png}
            \end{center}
          \end{minipage}
        \end{center}
        \caption[The count maps of PSR B1937+21 created from observation 
            data (left) and from the spectra model (right).]
            {The count maps of PSR B1937+21 created from observation 
            data (left) and from the spectra model (right). The dimensions
            of both figures are $141$ pixels $\times$ $141$ pixels and each pixel's size is
            $0.2^{\circ}\times0.2^{\circ}$.}
          \label{fig: j1939_count_map_ave}
        \end{figure}

        Figure \ref{fig: j1939_count_cube_ave} shows the count maps in different energy 
        bands. When energy is above about $1.6\mbox{GeV}$, there are very few photons. In 
        many cases, people analyze spectra from $100\mbox{MeV}$ to $300\mbox{GeV}$ with 
        \textit{Fermi} LAT, however, the number of photon counts above $100\mbox{GeV}$ is 
        nearly negligible compared with the total counts. The total number of photon counts 
        from $100\mbox{MeV}$ to $300\mbox{GeV}$ is about $1.07\times10^7$ compared with 
        $1175$ from $100\mbox{GeV}$ to $300\mbox{GeV}$. As a result, it is reasonable to 
        use data only from $100\mbox{MeV}$ to $100\mbox{GeV}$ and there should not be any 
        noticeable difference whether I use $100\mbox{GeV}$ to $100\mbox{GeV}$ or 
        $100\mbox{GeV}$ to $300\mbox{GeV}$. The case is also true for 
        the other two pulsars --- PSRs J0218+4232 and B1821$-$24. 

        \begin{figure}[!htp]
          \begin{minipage}{0.32\textwidth}
            \begin{center} 
              \includegraphics[scale=0.24]{j1939_ccube_1_ave.png}
            \end{center}
          \end{minipage}
          \begin{minipage}{0.32\textwidth}
            \begin{center}
              \includegraphics[scale=0.24]{j1939_ccube_13_ave.png}
            \end{center}
          \end{minipage}
          \begin{minipage}{0.32\textwidth}
            \begin{center}
            \includegraphics[scale=0.24]{j1939_ccube_19_ave.png}
            \end{center}
          \end{minipage}
          \caption[Three count maps of PSR B1937+21's count cube.]
            {Three count maps of PSR B1937+21's count cube. The energy ranges of the 
            figures are from $100$ to $131$ MeV, from $2460$ to $3212$ MeV, from $12198$ to
            $15929$ MeV respectively.}
          \label{fig: j1939_count_cube_ave}
        \end{figure}

      \subsubsection{Binned Likelihood Analysis}
        Figure \ref{fig: j1939_count_map_diff_ave} roughly describes how well the global fit 
        is. Both left and right panel are in linear scale in order to make it easier to 
        compare. As the residual map shows, there is no obvious difference between the fitted 
        model and the observational data. The max value of the photon counts in the residual map 
        is only about $270$. 
        \begin{figure}[!htp]
          \begin{center}
          \begin{minipage}{0.45\textwidth}
            \begin{center} 
              \includegraphics[scale=0.31]{j1939_cmap_linear_ave.png}
            \end{center}
          \end{minipage}
          \begin{minipage}{0.45\textwidth}
            \begin{center}
              \includegraphics[scale=0.365]{j1939_diff_map_linear_ave_new.png}
            \end{center}
          \end{minipage}
          \end{center}
            \caption[The count map created from observational data (left) and residual map 
              (right) of PSR B1937+21.]
              {The count map created from observational data (left) and residual map (right)
              of PSR B1937+21. Both left and right panels are in linear scale. 
              The dimension is $141$ pixels $\times 141$ pixels and each pixel's 
              size is $0.2^{\circ}\times0.2^{\circ}$.}
            \label{fig: j1939_count_map_diff_ave}
        \end{figure}

        Table \ref{table: j1939_fit_result_ave} lists the fit results of PLEC model. 
        According to the previous work \cite{0004-637X-787-2-167}), 
        the PLEC model is not preferred over the power-law model. Therefore, only 
        the results of power-law model are reported in the paper. In this result, the 
        PLEC model gives a TS value of 122, which is about the same as the previous 
        results. 

        \begin{table}[!ht]
          \centering
            \scalebox{0.8}{
            \begin{tabular}{|ccc|}
              \hline 
              & This Study & Previous Results \\
              \hline \hline 
              Photon Index ($\Gamma$) & $2.61\pm0.22$ & $2.1\pm0.2$ \\
              Cutoff Energy ($E_c$, GeV) & $4.90\pm2.29$ & $8\pm4$ \\
              % \hline 
              % Energy Flux ($10^{-11}$ erg $cm^{-2} s^{-1}$) & $4.61$ & $2.50$ & $~$ & $~$ \\ 
              \hline
            \end{tabular}}  
            \caption[Fit parameters of the spectra model of PSR B1937+21.]
              {Fit parameters of the spectra model of PSR B1937+21. 
              The names of parameters are consistent with Equation
              \ref{eq: fermi_model}. The old results are reported by
              \cite{0004-637X-787-2-167} Note that since the paper prefers power-law model 
              than PLEC model, it does not report the photon flux of PLEC model.}
            \label{table: j1939_fit_result_ave}        
        \end{table} 

        Figure \ref{fig: j1939_count_spectra_ave} shows the residual counts of the global fit.
        The fit is good from $100$MeV to about $20$GeV. However, it is significantly worse 
        when the energy is too high and the residual counts are much larger than the PSRs 
        J0218+4232 and B1821$-$24.

        Although the TS value for PSR B1937+21 is only 122, existence in gamma-ray of the 
        target source is more obvious in the TS maps as the Figure 
        \ref{fig: j1939_tsmap_comparison_15_ave} shows. Figure 
        \ref{fig: j1939_pl_and_cutoff_ave} contains the spectra shape of the PLEC model. 

        \begin{figure}[!htp]
          \centering
          \begin{minipage}{0.40\textwidth}
            \begin{center} 
              \includegraphics[scale=0.31]{j1939_tsmap_nosource_15_ave.png}
            \end{center}
          \end{minipage}
          \begin{minipage}{0.40\textwidth}
            \begin{center}
              \includegraphics[scale=0.31]{j1939_tsmap_withsource_15_ave.png}
            \end{center}
          \end{minipage}
          \caption[TS maps of PSR B1937+21.]
            {TS maps of PSR B1937+21. The dimension is $3^{\circ} \times 3^{\circ}$
            ($15$ pixels $\times 15$ pixels and each pixel is 
            $0.2^{\circ} \times 0.2^{\circ}$). The left panel is generated by the model 
            after removing PSR B1937+21 while the right panel contains the target source.}
          \label{fig: j1939_tsmap_comparison_15_ave}
        \end{figure}
  

        \begin{figure}[!htp]
          \centering 
          \includegraphics[scale=0.37]{what_j1939_count_spectra.png}
          \caption[The count spectra and count residuals of the fitted spectral model.]
            {Top: count spectra for all sources included in the xml model (including 
            sources with both free parameters and fixed parameters). Bottom: count residuals plot
            of the model. The thick red line is the target source PSR B1937+21, the blue thick 
            lines are sources with free parameters and the thin lines are the sources with only 
            fixed parameters.}
          \label{fig: j1939_count_spectra_ave}
        \end{figure}
            

        % \add{PLEC more, below is power-law model }

        Since the fit results of PLEC model is not very satisfiable, I also fit the 
        gamma-ray spectra with a power-law model. Because I have shown lots of count maps 
        and they actually do not give us very much information, I do not show more count maps 
        and count cubes for the analysis of the power-law model. The photon index is 
        $2.94\pm0.13$ with a TS value of $147$ as the Table 
        \ref{table: j1939_power_law_compare_ave} shows. Although the energy flux is consistent 
        with the previous study \cite{0004-637X-787-2-167}, the photon 
        index is not.
        \begin{table}[!htp]
          \centering
            \scalebox{0.8}{%
            \begin{tabular}{|ccc|}
              \hline
              & This Thesis & Previous Results \\
              \hline \hline 
              Photon Index ($\Gamma$) & $2.94\pm0.13$ & $2.38\pm0.07$  \\
              Energy Flux ($10^{-11}$ erg $\text{cm}^{-2} \text{s}^{-1}$) & $1.6\pm0.2$ & $1.6\pm0.2$   \\
              \hline 
            \end{tabular}}  
            \caption[Photon index comparison of power-law model between different 
              studies.]
              {Photon index comparison of power-law model between different 
              studies. The data of column Previous Results are from
              \cite{0004-637X-787-2-167}.}
            \label{table: j1939_power_law_compare_ave}        
        \end{table}  

        Figure \ref{fig: j1939_power_law_count_spectra} is the count residuals of the fit and 
        it is no better than Figure \ref{fig: j1939_count_spectra_ave}. But I also prefer a 
        power-law model since the likelihood of the power-law model is much larger than the 
        PLEC model. The Fermi tool \textit{gtlike} gives a likelihood value of the 
        fitting model. I compare the absolute value of the likelihood to decide which model 
        is preferred. The absolute value for the power-law model is $18409504.4$ compared with 
        $4427799.496$ for a PLEC model. And the spectrum of the power-law model is 
        shown in Figure \ref{fig: j1939_pl_and_cutoff_ave}. Both power-law and PLEC 
        models are consistent with the flux points, which are generated independently, and 
        the flux points do not show a clear cutoff. 

        \begin{figure}
          \centering 
          \includegraphics[scale=0.38]{j1939_power_law_count_spectra.png}
          \caption{Top panel is counts spectra for all sources within the ROI. Bottom panel  
            shows the count residuals. }
          \label{fig: j1939_power_law_count_spectra}
        \end{figure}

        Figure \ref{fig: j1939_power_law_tsmap_withsource_20} shows the TS maps with the 
        target source and without the target source. 

        \begin{figure}[!htp]
          \begin{center}
          \begin{minipage}{0.45\textwidth}
            \begin{center} 
              \includegraphics[scale=0.36]{j1939_power_law_tsmap_no_source_20.png}
            \end{center}
          \end{minipage}
          \begin{minipage}{0.45\textwidth}
            \begin{center}
              \includegraphics[scale=0.38]{j1939_power_law_tsmap_with_source_20_new.png}
            \end{center}
          \end{minipage}
        \end{center}

          \caption[TS maps of PSR B1937+21 (power-law model).]
            {TS maps of PSR B1937+21 (power-law model). The dimension is $4^{\circ} \times 4^{\circ}$
            ($20$ pixels $\times 20$ pixels and each pixel is 
            $0.2^{\circ} \times 0.2^{\circ}$). The left panel is generated by the model 
            after removing PSR B1937+21 while the right panel contains the target source.}
          \label{fig: j1939_power_law_tsmap_withsource_20}
        \end{figure}

        \begin{figure}[!htp]
          \centering
          \includegraphics[scale=0.3]{j1939_pl_and_cutoff_ave_deprecated.png}
          \caption[The log-log plot of flux to energy of PSR B1937+21’s gamma-ray spectrum.]
            {The log-log plot of flux to energy of PSR B1937+21’s gamma-ray spectrum.
            The red line represents the PLEC model and the green line is the power-law
            model. }
          \label{fig: j1939_pl_and_cutoff_ave}
        \end{figure}


    \subsection{Phase Resolved Analysis}
      Since the PLEC model is not a good model for phase-averaged spectrum of 
      PSR B1937+21, I did a quick phase-resolved analysis for the MPS. I choose the 
      phase of $0.0 \sim 0.2$ and $0.5 \sim 0.7$ from the full good time interval (GTI) as 
      Figure \ref{fig: j1939_phase} implies. 
      \begin{figure}[!htp]
        \centering 
        \includegraphics[scale=0.55]{b1937_parfiles.png}
        \caption{Top left panel shows the pulse phase of PSR B1937+21 in gamma-ray band. 
          The figure is produced by tempo2.}
        \label{fig: j1939_phase}
      \end{figure}

      The light curve generated by \textit{Fermi} LAT is consistent with previous results 
      \cite{J1939_old} as Figure \ref{fig: j1939_light_curve_compare} shows. 
      The new light curve by LAT is better than the previous LAT light curve
      as shown in the figure. 

      \begin{figure}
        \centering 
        \includegraphics[scale=0.4]{j1939_lightcurve_all.png}
        \caption{Light curves of PSR B1937+21 by different telescopes. The data and figure
          are from \cite{J1939_old}}
        \label{fig: j1939_light_curve_compare}
      \end{figure}


      The radius of the ROI is also set to $20^{\circ}$ degrees, and all parameters of 
      sources $8^{\circ}$ degrees outside from the center are fixed with default values, 
      which is the same as the above analysis. 
      Figure \ref{fig: j1939_count_map} is the comparison of count maps between 
      observational data and the model.

      \subsubsection{Count Maps and Count Cubes}
        \begin{figure}[!ht]
          \begin{center}
          \begin{minipage}{0.45\textwidth}
            \begin{center} 
              \includegraphics[scale=0.38]{j1939_cmap.png}
            \end{center}
          \end{minipage}
          \begin{minipage}{0.45\textwidth}
            \begin{center}
              \includegraphics[scale=0.38]{j1939_cmap_model.png}
            \end{center}
          \end{minipage}
          \end{center}
          \caption[The count maps of PSR B1937+21 created from observation 
                  data (left) and from the spectra model (right).]
              {The count maps of PSR B1937+21 created from observation 
              data (left) and from the spectra model (right). The dimensions
              of both figures are $141$ pixels $\times 141$ pixels and each pixel's size is
              $0.2^{\circ}\times0.2^{\circ}$.}
            \label{fig: j1939_count_map}
        \end{figure}

        There are four point sources with free parameters in the model which are represented 
        by the green circles in Figure \ref{fig: j1939_count_map} and I add the PLEC model 
        for PSR B1937+21 manually. 

        Like the phase-averaged case, the count map is so messy that the source PSR B1937+21 
        is completely not identifiable in this count map.  
        Figure \ref{fig: j1939_count_cube} shows count maps in different energy bands. As 
        the figure shows, the lower energy band is very messy while there is no valuable  
        data in the high energy band. Therefore, the data in the middle energy range is 
        more reliable and when fitting the two-layer model, it has higher priority to 
        minimize the differences between the model and observational data in the middle part.

        \begin{figure}[!htp]
          \begin{minipage}{0.32\textwidth}
            \begin{center} 
              \includegraphics[scale=0.24]{j1939_ccube_start.png}
            \end{center}
          \end{minipage}
          \begin{minipage}{0.32\textwidth}
            \begin{center}
              \includegraphics[scale=0.24]{j1939_ccube_middle.png}
            \end{center}
          \end{minipage}
          \begin{minipage}{0.32\textwidth}
            \begin{center}
            \includegraphics[scale=0.24]{j1939_ccube_end.png}
            \end{center}
          \end{minipage}
          \caption[Three count maps extracted from PSR B1937+21's count cube.]
            {Three count maps extracted from PSR B1937+21's count cube. The energy ranges 
            of the figures are from $100$ to $126$ MeV, from $1.99$ to $2.51$ GeV, from $50.12$ 
            to $63.10$ GeV respectively.}
          \label{fig: j1939_count_cube}
        \end{figure}

      \subsubsection{Binned Likelihood Analysis}
        Figure \ref{fig: j1939_count_map_diff} is a combination of the residual map the 
        original count map of PSR B1937+21 in linear scale. The fact that there are many 
        dots in the residual map implies that the global fit is not as good as the 
        other two pulsars. Furthermore, the fit is particularly bad in the cyan circle 
        region in the right bottom part Figure \ref{fig: j1939_count_map_diff}. 
        The total numbers of counts of the cyan region are $18975$ and $4212$ respectively 
        for the count map and the residual map. The center of the region is 
        $\left(290.53^{\circ}, 14.12^{\circ}\right)$ and is about $8.65^{\circ}$ away from 
        the source, which means that spectral parameters of the sources in the region are 
        fixed during the fit. Therefore, this may be because that the default values from 
        \textit{Fermi} LAT four-year Point Source Catalog (3FGL) in this region 
        are not consistent with the observational data. 
        % during the fit. This may be the problem of the method of choosing the phase. 
        % Firstly, I get the pulse phase of PSR B1937+21 and select $40\%$ of the whole 
        % rotation period containing the pulse phase. At the same time, I need to change the 
        % scales to $40\%$ of their original values for the sources with no free 
        % parameters. This is reasonable, however, for some bright sources, a little 
        % difference in percent can cause big residuals of photon counts. 
        % \question{Not Sure}

        \begin{figure}[!htp]
          \begin{center}
          \begin{minipage}{0.45\textwidth}
            \begin{center} 
              \includegraphics[scale=0.31]{j1939_cmap_linear.png}
            \end{center}
          \end{minipage}
          \begin{minipage}{0.45\textwidth}
            \begin{center}
              \includegraphics[scale=0.34]{j1939_cmap_diff_linear_new.png}
            \end{center}
          \end{minipage}
          \end{center}

          \caption[The count maps of PSR B1937+21 created from observation 
                  data (left) and from the spectra model (right).]
            {The count maps of PSR B1937+21 created from observation 
            data (left) and from the spectra model (right). The dimensions
            of both figures are $141$ pixels $\times 141$ pixels and each pixel's size is
            $0.2^{\circ}\times0.2^{\circ}$.}
          \label{fig: j1939_count_map_diff}
        \end{figure}

        The Figure \ref{fig: j1939_count_spectra} is the combination of count spectra for 
        all sources within the $20^{\circ}$ and the count residual plot. The count residual 
        plot shows that except low gamma-ray energy part (from $100\hbox{MeV}$ to 
        $200\hbox{MeV}$) and very high energy band (above 10GeV) the total number of photon 
        counts of the model is very close to the observational data. This seems good but 
        also implies some problems. As previously discussed, the modeled number of photon counts 
        in the cyan circle is $7924$ smaller than the observational data. And the count residual 
        plot shows that the count number of the whole map is nearly the same, which means that 
        in the model there must be some sources forced to be larger than the real value to make 
        up for the deficiency. This implies that the fit results of some sources with free 
        parameters are not as good as the Figure \ref{fig: j1939_count_spectra} shows. 
        
        \begin{figure}[!htp]
          \centering
          \includegraphics[scale=0.42]{j1939_count_spectra.png}
          \caption{The count spectra (top) of all sources including those with only fixed 
            parameters and count residuals (bottom) of the fit for PSR B1937+21.}
          \label{fig: j1939_count_spectra}
        \end{figure}

        Table \ref{table: j1939_fit_result} lists the results of the fit parameters of 
        PSR B1937+21. We see from the table that the new fit results are 
        consistent with the old ones and the precision improves a lot. 
        Figure \ref{fig: j1939_pl_and_cutoff} is a plot of the PLEC model.
        The global fit is consistent with flux points fitted by each energy bin separately 
        as the flux points and the red line with shade shows. Surprisingly, the power-law model
        is very close to the phase-resolved PLEC model. 
        The TS value of our target source is $383$, which gives 
        a significance level $\sigma \approx \sqrt{TS} = 19.6$. This implies the presence 
        of the target source. I also generate TS maps to test the presence of the source 
        as Figure \ref{fig: j1939_tsmap_comparison_15} shows. 
        
        \begin{figure}[!htp]
          \centering 
          \includegraphics[scale=0.4]{j1939_pl_and_cutoff.png}
          \caption{PLEC model spectrum of PSR B1937+21 (phase resolved). }
          \label{fig: j1939_pl_and_cutoff}
        \end{figure}
          
        \begin{figure}[!htp]
          \centering
          \begin{minipage}{0.40\textwidth}
            \begin{center} 
              \includegraphics[scale=0.31]{j1939_tsmap_nosource_15.png}
            \end{center}
          \end{minipage}
          \begin{minipage}{0.40\textwidth}
            \begin{center}
              \includegraphics[scale=0.31]{j1939_tsmap_withsource_15.png}
            \end{center}
          \end{minipage}
          \caption[TS maps of PSR B1937+21.]
            {TS maps of PSR B1937+21. The dimension is $3^{\circ} \times 3^{\circ}$
            ($15$ pixels $\times 15$ pixels and each pixel is 
            $0.2^{\circ} \times 0.2^{\circ}$). The left panel is generated by the model 
            after removing PSR B1937+21 while the right panel contains the target source.}
          \label{fig: j1939_tsmap_comparison_15}
        \end{figure}

        \begin{table}[!htp]
          \centering
            \scalebox{0.8}{
            \begin{tabular}{|ccc|}
              \hline 
              & This Study & Previous Results \\
              \hline \hline
              Photon Index ($\Gamma$) & $2.37\pm0.06$ & $1.43\pm0.87$ \\
              Cutoff Energy($E_c$, GeV) & $4.5\pm1.1$ & $1.15\pm0.74$ \\
              % \hline 
              % Energy Flux (erg $cm^{-2} s^{-1}$) ($10^{-12}$) & $19$ & $0.3$ & $16$ & $2$ \\ 
              \hline
            \end{tabular}}  
            \caption[Fit parameters of the spectra model of PSR B1937+21 (phase resolved).]
              {Fit parameters of the spectra model of PSR B1937+21 (phase resolved). 
              The previous results are from \cite{0004-637X-787-2-167}.
              Since the studied energy range is different, I do not list the energy flux for 
              comparison.}
            \label{table: j1939_fit_result}        
        \end{table} 

  \section{Summary of the Results}
    The spectra of PSRs J0218+4232, B1937+21 and B1821$-$24 are studied using more observation 
    data and newly published \textit{Pass 8} dataset. The new results of the thesis are 
    generally consistent with the old results as the Table \ref{table: final_results} shows. 
    For PSR B1937+21, I also list the results of power-law model as Table 
    \ref{table: final_results_j1939_pl} shows.
    \begin{table}[!htp]
      \centering
        \resizebox{\textwidth}{!}{%
        \begin{tabular}{|c|c|c|c|c|c|c|}
          \cline{1-7}
          & \multicolumn{2}{|c|}{J0218+2134} & \multicolumn{2}{|c|}{B1937+21} & \multicolumn{2}{|c|}{B1821$-$24} \\
          \cline{1-7} 
          & New & Previous & New & Previous & New & Previous \\
          \cline{1-7}
          Photon Index ($\Gamma$) & $1.89\pm 0.04$ & $2.0\pm0.1$ & $2.37\pm0.06$ & $2.1\pm0.2$ & $1.91\pm0.07$ & $1.6\pm0.3$  \\
          Cutoff Energy (GeV)& $3.77\pm0.40$ & $4.6\pm1.2$ & $4.5\pm1.1$ & $8\pm4$ & $4.50\pm0.71$ & $3.3\pm1.5$\\
          Photon Flux ($10^{-8}$ $\text{cm}^{-2} \text{s}^{-1}$) &$7.29\pm0.28$ &$7.7\pm0.7$ &$3.76\pm0.35$ & ~ &$3.85\pm0.31$ & $1.5\pm0.6$ \\
          Energy Flux ($10^{-11}$ erg $\text{cm}^{-2} \text{s}^{-1}$) & $4.45\pm0.16$ & $4.56\pm0.24$ & $1.58\pm0.15$ & ~ & $2.44\pm0.14$ & $1.3\pm0.2$ \\
          \hline 
        \end{tabular}}  
        \caption[Fit parameters of the spectra model of PSR J0218+4232.]
          {Fit parameters of the spectra model of PSR J0218+4232. 
          The names of parameters are consistent with Equation
          \ref{eq: fermi_model}. The previous results are from  
          \cite{0004-637X-787-2-167,0067-0049-208-2-17}}
        \label{table: final_results}        

    \end{table}  

    \begin{table}[!htp]
      \centering 
      \scalebox{0.8}{
        \begin{tabular}{|ccc|}
        \hline 
        & New Results & Previous Results \\ 
        \hline \hline 
        Photon Index ($\Gamma$) &  $2.94\pm0.13$ & $2.38\pm0.07$ \\
        Photon Flux ($10^{-8}$ $\text{cm}^{-2} \text{s}^{-1}$) & $5.89\pm0.68$ & $2.93$ \\
        Energy Flux ($10^{-11}$ erg $\text{cm}^{-2} \text{s}^{-1}$) & $1.92\pm0.18$ & $ 1.6\pm0.2$ \\ 
        TS value & 147 & 112 \\
        \hline 
      \end{tabular}}
      \caption[Fit results of power-law model for PSR B1937+21.]
        {Fit results of power-law model for PSR B1937+21. Since the previous paper 
        does not show the photon flux, I cannot calculate the error bars of photon flux for 
        the previous value.}
      \label{table: final_results_j1939_pl}
    \end{table}

    With \textit{Pass 8} data and more observational data, the new results have smaller 
    error bars in terms of photon index and energy flux. Furthermore, we have much larger 
    TS values for PSRs J0218+4232 and B1821$-$24 as the Table \ref{table: ts_value_compare} 
    shows. 
    
    % For PSR B1937+21, a power-law model should be preferred because of the 
    % following reasons. The $-log(likelihood)$ of PLEC model and power-law model are 
    % $-4410944$ and $-18409504$ respectively, the 
    % $TS_{cut} = 2\Delta log(likelihood) = -2.86 < 9$,  thus I follow the convention of the 
    % paper \cite{2013ApJS..208...17A}.

    \begin{table}[!htp]
      \centering
        \scalebox{0.8}{
        \begin{tabular}{|cccc|}
          \hline 
          & J0218+4232 & B1821$-$24 & B1937+21 \\ 
          \hline 
          Previous & 1313 & 76 & 112 \\
          New Results & 6809 & 941 & 147 \\
          \hline 
        \end{tabular}}  
        \mycaption{TS values comparison between the new results and the previous results for 
          the three MSPs. }
        \label{table: ts_value_compare}        
    \end{table}  

    
  
        
\chapter{Theory and Simulation}
  \section{A Brief Introduction to the Outer Gap Model}
    It is oversimplified to regard a pulsar as a magnetized sphere rotating in a vacuum. 
    Actually, there are plenty of charged particles in a pulsar's magnetosphere 
    which co-rotate with the pulsar. The creation of charged particles can 
    be described following steps \cite{Sturrock:1971zc}.

    % The co-rotating charged primary particles emit gamma-ray by \\curvature 
    % radiation because they are accelerated by environmental magnetic fields.  
    In the intense magnetic field, the high energy photons decay into electrons and 
    positrons which are called secondary particles by the process: 
    $\gamma + (B) \rightarrow e^++e^-+(B)$ and these charged particles can emit 
    synchrotron radiation. The secondary particles in charge-deficient regions can also 
    be accelerated to very high speed by strong magnetic field just like primary particles 
    and some of them then emit gamma-rays which can further decay into electrons and 
    positrons. As a result, these charged particles can create more secondary particles.
    This chain of the processes is quite efficient to produce charged particles and 
    make a pulsar's magnetosphere filled with plasma as a consequence. Therefore, a 
    characteristic charge density $\rho_{GJ}=-\frac{\vec{\Omega}\cdot \vec{B}}{2\pi c}$ 
    called Goldreich-Julian charge density is produced \cite{1969ApJ}.
    By the definition of $\rho_{GJ}$, there is a surface called null charge surface where 
    $\rho_{GJ}$ is very close or equal to $0$. 

    Since the charged particles cannot move along the magnetic field lines near 
    the light cylinder (they cannot exceed the speed of light), there are closed field 
    lines and open field lines. Charged particles can move out of the magnetosphere along 
    these open field lines. Thus, the charge densities in the regions near the light cylinder
    can be much smaller than the \gj{} and electrons and positrons are accelerated to very high 
    speeds by electric fields parallel to the magnetic field lines ($E_{\parallel}$). 
    These regions are called the outer gap which is between the null charge surface and the 
    light cylinder \cite{1986ApJ...300..500C}.
    
    Since electrons and positrons are accelerated to opposite directions, there are many 
    charged particles move toward the pulsar's surface. As discussed above, there is no 
    significant electric field to accelerate the incoming particles, they emit softer 
    photons than the photons emitted by the particles moving outward. Furthermore, since the 
    softer photons are close to the stellar surface, they can generate electrons and 
    positrons with the help of high magnetic fields. These charged particles maintain the current 
    moving along the open field lines. Meanwhile, the particles moving from the null charge 
    surface to the light cylinder are largely accelerated by $E_{\parallel}$, and hence 
    emit gamma-rays by curvature radiation. Part of the gamma-rays covert to
    electron-positron pairs by colliding with soft photons, and the pairs compensate 
    the deficit of charge densities, hence stop the growth of the outer gap. This is the 
    very basic introduction of the outer gap model, which is helpful to understand the 
    two-layer model. 
        
  \section{Two-Layer Model}
    After reviewing the gamma-ray emission mechanism, we can proceed to
    the two-layer model on which this thesis is mainly based
    \cite{0004-637X-720-1-178}. The two-layer model is a variation of the outer-gap model 
    since they both claim that the gamma-ray emission zone is close to the light cylinder. 
    However, in the two-layer model, the outer layer consists of two regions --- a primary 
    acceleration region and a screening region. 

    In the primary region, charged particles moved out of pulsars along the open field 
    lines, so the charge density is usually very low. 
    However, by pair-production processes, a lot of $e^{-}$ and $e^{+}$ are produced. 
    When lots of pairs are created initially, the charge density 
    doesn't change very much because the pairs have not been separated yet. With the help 
    of the strong electric field, the particles of opposite signs move to opposite directions. 
    As a result, the two-layer model states that above the primary region, a screening 
    region will be created and the charge density is very large because of the accumulation 
    of the charged particles. This is basically the reason why there are two regions in 
    pulsars' outer magnetosphere.

    The next issue is that how to describe the distribution of charge densities in these two 
    regions. For simplicity, the model just uses a step function to represent the charge 
    density distribution and a step function can well describe the large charge density 
    difference between the two regions. It uses a magnetic dipole model to approximate the 
    magnetic distribution in the magnetosphere. Since by magnetic dipole model, the magnetic 
    field at one position is only dependent on the position's distance from the source 
    and altitude, the model also ignores the azimuthal distribution of charge density 
    and uses two parameters which are distance $r$ and altitude $\theta$ to calculate the 
    magnetic field at a particular position.

    The two-layer model uses three parameters to express the structure of a pulsar's outer 
    magnetosphere --- charge density of the primary region, the total height of the primary 
    region and the screening region and the last one is the ratio of the thickness of the 
    primary region and the screening region. Figure \ref{fig: charge_density} shows the 
    basic structure of two-layer model. 

    \begin{figure}[!htp]
      \centering 
      \includegraphics[scale=0.6]{charge_density.png}
      \caption[Geometry and charge density distribution of the two-layer model.]
        {(a): Geometry of the two-layer model. $h_{1}$ and 
        $h_{2}$ is the height of the primary region and the screening region respectively. 
        (b): the charge density distribution of the primary region and the screening region. 
        In the primary region, the charge density is much smaller than Goldreich-Julian charge 
        density while is larger in the screening region. The figure is reported by
        \cite{0004-637X-720-1-178}.}
      \label{fig: charge_density}
    \end{figure}
    % \singleFig{charge_density}{0.6}{(a): Geometry of the two-layer model. $h_{1}$ and 
    %   $h_{2}$ is the height of the primary region and the screening region respectively. 
    %   (b): the charge density distribution of the primary region and the screening region. 
    %   In the primary region, the charge density is much smaller than Goldreich-Julian charge 
    %   density while is larger in the screening region. The figure is from. 
    %   \cite{0004-637X-720-1-178}}

    As Figure \ref{fig: charge_density} shows, 
    let the charge density of the primary region be $\rho_1 = (1-g_{1}) \rho_{GJ}$ and 
    the total gap size be
    $h_{2}$, where $\rho_{GJ}$ is the Goldreich-Julian charge density. For convenience, 
    also denote the gap size of the primary region as $h_{1}$. 
    \myComment{Then we can calculate electric potential and electric field by solving the 
    Poisson equation }
    Denote the electrical potential as $\phi_{0}$ which satisfies 
    \begin{equation}
      \label{eq: Poisson_corotating}
      \nabla^{2}\phi_{0} = -4\pi\rho_{GJ} ,
    \end{equation}
    and the total electrical potential is $\phi = \phi_{0} + \phi^{\prime}$, 
    where $\phi^{\prime}$ is a representation of the deviation from the co-rotating 
    electrical potential. Let the total charge density be $\rho$ and subtract by
    Equation \ref{eq: Poisson_corotating} we have,
    \begin{equation}
      \label{eq: Poisson_final}
      \nabla^{2}\phi^{\prime} = -4\pi\left(\rho - \rho_{GJ} \right) .
    \end{equation}

    Because the model has ignored the distribution in the azimuthal direction, it uses two 
    parameters $x, z$ to represent a position, where $x$ is the direction along the magnetic 
    field line and $z$ is perpendicular to the magnetic field line. In order to solve 
    Equation \ref{eq: Poisson_final}, the model also makes two approximations. The first is that 
    the derivative of electrical potential $\phi$ is ignored. The second is that the \gj{} 
    is uniformly distributed along the magnetic line direction ($x$ direction). These two 
    approximations are based on a reasonable assumption that the change rate 
    for both electrical potential ($\phi^{\prime}$) and \gj{}($\rho_{GJ}$) along the $x$ 
    direction is much smaller than the $z$ direction. 
    As a consequence, Equation \ref{eq: Poisson_final} can be written as: 
    \begin{equation}
      \label{eq: Poisson_final_final}
      \frac{\partial^2}{\partial z^2} \phi^{\prime} = -4\pi\left(\rho - \rho_{GJ} \right) .
    \end{equation}

    In order to solve Equation \ref{eq: Poisson_final_final}, proper boundary conditions are also 
    needed. First of all, we have to decide the boundary positions, which is determined by 
    four parameters and they can be written as $x_{lo}, x_{hi}, z_{lo}, z_{hi}$. It is 
    reasonable to set $x_{lo}$ and $x_{hi}$ be the stellar surface and the light cylinder 
    respectively and $z_{lo}$ (lower boundary) be the last open field line. And let the 
    electrical potential be $0$ along the last open field line (this is because the 
    variation of electric field strength along the $x$ direction is ignored) as Equation 
    \ref{eq: lower_boundary} shows.  

    \begin{equation}
      \label{eq: lower_boundary}
      \phi \left(x, z_{lo}\right) = 0 .
    \end{equation}
    To determine the position of $z_{hi}$ is a little bit tricky. In order to make the electrical 
    potential be continuous at $z = z_{hi} = h_2$, the model sets the 
    $\phi^{\prime}\left(z=h_{2}\right) = 0$ since the non-co-rotating electrical potential 
    outside the upper bound is $0$ and the co-rotating potential is continuous near the 
    boundary. Additionally, because $\phi^{\prime}\left(z=h_{2}-\right) = 0$
    and $\phi^{\prime}\left(z=h_{2}+\right) = 0$, it is known that the first derivative 
    $\partial{\phi^{\prime}}/\partial{z}\vert_{z=h_{2}}$ is $0$, which means 
    $E_{\perp}\vert_{z=h_{2}} = 0$. In order to solve Equation \ref{eq: Poisson_final_final}, 
    denote charge densities of the two regions for convenience as the function 
    \ref{eq: twolayer_charge_density} shows.
    \begin{equation}
      \label{eq: twolayer_charge_density}
        \rho\left(z\right) = 
        \begin{cases}
          & \rho_{1} , \text{    if} \left(0 \leq z < h_{1}\right)\\
          & \rho_{2} , \text{    if} \left(h_{1} \leq z \leq h_{2}\right) .
        \end{cases}       
    \end{equation}
    With definition in Equation \ref{eq: twolayer_charge_density} and the three boundary 
    conditions, the solution of Equation \ref{eq: Poisson_final_final} is, 
    \begin{equation}
      \label{eq: twolayer_potential}
        \phi^{\prime}\left(z, x\right) = -2\pi
        \left\{\begin{alignedat}{2}
          & \left(\rho_{1} - \rho_{GJ}\left(x\right)\right)z^2 + C_{1} z ,  &&\left(0 \leq z < h_{1}\right)\\
          & \left(\rho_{2}-\rho_{GJ}\left(x\right)\right)\left(z^2 - h_2^2\right) + D_{1} \left(z-h_2\right),  &&\qquad \left(h_{1} \leq z \leq h_{2}\right) ,
        \end{alignedat}\right.
    \end{equation}
    where 
    \begin{equation*}
      C_{1} = \frac{\left(\rho_{1}-\rho_{GJ}\left(x\right)\right)h_1\left(h_1-2h_2\right)-\left(\rho_2-\rho_{GJ}\left(x\right)\right)\left(h_1-h_2\right)^2}{h_2} ,
    \end{equation*}
    and 
    \begin{equation*}
      D_{2} = \frac{\left(\rho_1-\rho_2\right)h_1^2-\left[\rho_2-\rho_{GJ}\left(x\right)\right]h_2^2}{h_2} .
    \end{equation*}
    From Equation \ref{eq: twolayer_potential} and apply 
    $\rho_{GJ}\left(x\right)=-\left(\Omega B x\right)/\left(2\pi cs\right)$ 
    we can directly derive the strength of electrical fields parallel to magnetic field lines as a 
    function of $z$ as Equation \ref{eq: twolayer_field} shows.
    \begin{equation}
      \label{eq: twolayer_field}
        E^{\prime}_{\parallel}\left(z\right) = \frac{\Omega B}{cs}
        \left\{\begin{alignedat}{2}
          & -g_1 z^2 + C_1^{\prime}z ,  &&\left(0 \leq z < h_{1}\right) \\
          & g_2\left(z^2 - h_2^2\right) + D_1^{\prime}\left(z-h_2\right)  &&\qquad \left(h_{1} \leq z \leq h_{2}\right) .
        \end{alignedat}\right.
    \end{equation}
    where 
    \begin{equation*}
      C_{1}^{\prime} = \frac{g_1 h_1 \left(h_1 - 2h_2\right)+ g_2\left(h_1-h_2\right)^2}{h_2} ,
    \end{equation*}
    and 
    \begin{equation*}
      D_{2}^{\prime} = -\frac{\left(g_1 + g_2\right)h_2^2 + g_2 h_2^2}{h_2} .
    \end{equation*}

    Since charged particles are accelerated in the primary region to relativistic speeds 
    and then emit curvature radiation, we have
    \begin{equation}
      \label{eq: all_is_curvature_radiation}
      e E_{\parallel}^{\prime} c = l_{cur} ,
    \end{equation}
    where $E_\parallel^{\prime}$ is the electric field strength along the magnetic field 
    line described in Equation \ref{eq: twolayer_field}.
    Lorentz factors of the charged particles is estimated according to Equation 
    \ref{eq: curvature_radiation_power}.
    \begin{equation}
      \label{eq: curvature_radiation_power}
      l_{cur} = \frac{2 e^2 c \gamma^{4}_{e}}{3s^2} ,
    \end{equation}      
    where $s$ is the radius of curvature. 
    Combining Equations \ref{eq: all_is_curvature_radiation} and 
    \ref{eq: curvature_radiation_power} we get expression for Lorentz factor: 
    \begin{equation}
      \label{eq: gamma_can_be_zero}
      \gamma_{e} = \left(\frac{3s^2}{2e} E_{\parallel}^{\prime}\right)^{1/4} .
    \end{equation}
    With the Lorentz factor of a charge particle known, we can derive the curvature radiation 
    spectrum for a single charged particle and then can get the total spectrum by integrating 
    over all charged particles. This is the simplified idea of the two-layer model. 
  

  \section{X-ray Emission Model}
    The three MSPs have very hard photon indices which is about one as Table 
    \ref{table: X-ray_photon_indices} shows. The hard photon indices can be explained by 
    inverse-Compton scattering between the radio waves and the primary particles in the 
    outer magnetosphere. The hard X-ray flux is contributed by both inverse-Compton 
    scattering and synchrotron radiation emitted by the secondary particles. 
    
    Firstly, we consider inverse-Compton scattering. \cite{0004-637X-787-2-167} uses a 
    broken power-law model with a turnover at 10MHz to describe the radio spectrum as 
    Function \ref{eq: radio_spectrum} shows \cite{0004-637X-787-2-167}.
    \begin{equation}
      \label{eq: radio_spectrum}
        F_{rad}\left(\nu\right) = A \left.
        \begin{cases}
          \left(\frac{\nu}{100\text{MHz}}\right)^{\beta_1} & \text{for } \nu \geq 10\text{MHz} \\
          \left(\frac{10\text{MHz}}{100\text{MHz}}\right)^{\beta_1} \left(\frac{\nu}{10\text{MHz}}\right)^{\beta_2}& \text{for } \nu > 10\text{MHz} .
        \end{cases}     
        \right.  
    \end{equation} 
    The spectral index $\beta_2$ is fixed to be $0.5$ and $\beta_1$ is estimated by the 
    flux densities at 1.4GHz ($\mbox{F}_{1400}$) and 400MHz ($\mbox{F}_{400})$. The data 
    used in the thesis are listed in Table \ref{table: radio_data_atnf}. 

    By approximating the magnetic field lines as concentric circles, the collision angle 
    between radio waves and primary particles can be estimated by 
    $\sin{\theta_0}\sim \sqrt{2f}$, where $f$ is the fractional gap size which is the 
    ratio of the total gap size to the length of the light cylinder. According to 
    \cite{0004-637X-787-2-167}, inverse-Compton radiation power per unit energy per solid 
    angle can be estimated by Function \ref{eq: IC_theory},
    \begin{equation}
      \label{eq: IC_theory}
      \frac{dP_\text{IC}}{d\Omega} = \mathcal{D}^{2}\left(1-\beta \cos{\theta_{0}}\right) F_{rad} \frac{d\sigma^{\prime}}{d\Omega^{\prime}} ,
    \end{equation}
    where  $\mathcal{D} = \gamma^{-1}\left(1-\beta\cos{\theta_1}
    \right)^{-1}$, $\theta_1$ is the angle between the scattered photon and the primary particle, 
    $F_{rad}$ is the radio spectrum 
    as discussed above, $d\sigma^{\prime}/d\Omega^{\prime}$ is the Klein-Nishina cross 
    section in electron-rest frame.
    Therefore, with the Functions \ref{eq: IC_theory} and \ref{eq: radio_spectrum}, the 
    flux of inverse-Compton radiation can be calculated. 

    Secondly, secondary particles can also emit non-thermal hard X-rays. Although it may 
    not be as strong as inverse-Compton scattering in very hard X-ray band, it should also 
    be included. As previous introduced, some particles are accelerated by the electric 
    fields parallel to the magnetic field lines ($E_{\parallel}$) and move toward the 
    stelar surface. These particles can heat up the polar cap region to about 1MK
    \cite{0004-637X-745-1-100}. Then new electron-positron pairs can be created by collisions
    between the thermal X-rays and gamma-rays. The optical depth can be calculated by 
    Function \ref{eq: X-ray_optical_depth}, 
    \begin{equation}
      \label{eq: X-ray_optical_depth}
      \tau_{X_{\gamma}} = \frac{L_{t} \sigma_{X_{\gamma}} R_{\text{lc}}}{4\pi R_{\text{lc}}^2 ck_{B}T} ,
    \end{equation}
    where $T$ is the surface temperature and
    $L_{t}$ is the thermal radiation, $\sigma_{X_{\gamma}} \sim \sigma_T / 3$ and 
    $k_B$ is the Boltzmann constant. Then the distribution of the secondary pairs can be 
    described by Function \ref{eq: X-ray_pair_distribution},
    \begin{equation}
      \label{eq: X-ray_pair_distribution}
      \frac{dN}{d\gamma_s} \left(\gamma_s\right) = \frac{m_e c^2}{\dot{E}_{sync}}\int_{2\gamma_s m_e c^2}^{\infty} Q\left(E_{\gamma}^{\prime}\right) dE_{\gamma}^{\prime} ,
    \end{equation}
    where $\gamma_s$ is the Lorentz
    factor of the secondary particles, $\dot{E}_{sync}$ is the energy loss rate of the 
    synchrotron radiation. According to \cite{0004-637X-787-2-167}, the synchrotron 
    spectrum can be calculated by Function \ref{eq: X-ray_spectrum_final},
    \begin{equation}
      \label{eq: X-ray_spectrum_final}
      F_{sync}\left(E_{\gamma}\right) = \frac{\sqrt{3}e^3B \sin{\theta_s}}{h m_e c^2} \int\frac{dN}{d\gamma_s} F\left(\frac{E_{\gamma}}{E_{sync}}\right)d\gamma_s ,
    \end{equation}
    where $E_{sync} = 3he\gamma_s^2B \sin{\theta_s}/4\pi m_e c$ and 
    $\sin{\theta_s}\sim\sqrt{2f}$, which is like the approximation in the inverse-Compton 
    scattering model.

    Add the synchrotron radiation and inverse-Compton scattering together, we can calculate
    the flux in hard X-ray and gamma-ray bands. In addition, by adding the observational data
    from \textit{NuSTAR} \cite{0004-637X-845-2-159}, we generate broadband spectra for all
    the three MSPs.

    \begin{table}[!htp]
      \centering
      \scalebox{0.8}{
        \begin{tabular}{|cccc|}
          \hline 
          & J0218+4232 & B1937+21 & B1821$-$24 \\
          \hline \hline 
          Photon Index ($\Gamma_{X}$) & $1.10\pm0.06$ & $0.9\pm0.1$ & $1.23\pm0.03$ \\
          Luminosity ($L_{X}, 10^{32}$ erg $\mbox{s}^{-1}$) & $3.3$ & $6.8$ & $14$ \\
          Pulse Fraction (\%) & $64\pm6$ & $\sim100$ & $82.5\pm4$ \\
          \hline 
        \end{tabular}}
        \caption[X-ray Properties of the PSRs J0218+4232, B1937 and B1821$-$24.]
          {X-ray Properties of the PSRs J0218+4232, B1937 and B1821$-$24. The data are from 
          \cite{2011ApJ...730...81B,2004A&A...417..181W,2002ApJ...577..917K,2012ApJ...760...92H,
          2010arXiv1006.0335B}}
        \label{table: X-ray_photon_indices}
    \end{table}
    
    \begin{table}[!htp]
      \centering
      \scalebox{0.8}{
        \begin{tabular}{|cccc|}
          \hline 
          & J0218+4232 & B1937+21 & B1821$-$24 \\
          \hline \hline 
          $\mbox{F}_{400}$ (mJy) & $35\pm5$ & $40$ & $240$ \\
          $\mbox{F}_{1400}$ (mJy) & $0.9\pm0.2$ & $2.0\pm0.4$ & $13.2\pm5$ \\
          \hline 
        \end{tabular}}
        \caption[Flux densities of PSRs J0218+4232, B1937+21 and B1821$-$24.]
          {Flux densities of PSRs J0218+4232, B1937+21 and B1821$-$24. The data are 
          from the ATNF catalogue. \footnote{http://www.atnf.csiro.au/research/pulsar/psrcat/}}
        \label{table: radio_data_atnf}
    \end{table}
  \section{Simple Optimizations of Numerical Calculation}
    \subsection{Accuracy}
      To make sure the numerical computation be right is the most important. 
      The first consideration is underflow and overflow of floating digits.
      One possible condition is calculating speeds of relativistic charged particles with 
      Lorentz factor $\gamma$. By doing some simple test, I find that for 
      $\gamma < 1.5\times 10^7$, the results are precise enough. However, there are 
      significant rounding errors when $\gamma > 5\times 10^7$, which means that the 
      results might be wrong for highly energetic particles if I directly use the formula 
      $\beta = \sqrt{1 - 1/\gamma^2}$.
      Likely, in the two-layer model, nearly all particles have $\gamma < 1\times 10^7$. 
      Furthermore, there are nearly no situations where double precision floating digits
      cannot handle calculation results of the two-layer model. Thus, as long as using 
      64-bit floating digits instead of 32-bit floating digits, we are free from overflowing 
      and underflow errors. 
        
      There are some cases when a whole function can be calculated while some parts of them 
      are not. Take Function $f\left(x\right) = x\times1/x$ for example. When $x$ is too 
      large, it can not be expressed by a computer and multiplication is not associative when 
      doing floating point operation. I encounter some situations like this.
      The formula of curvature radiation spectrum contains modified Bessel function of order 
      $5/3$. In order to speed up the program, I use a polynomial to express the Bessel 
      function, as Equation \ref{func: polynomial_appro} shows. 
      \begin{equation}
        K_{5/3} \left(x\right) \simeq a \left(\frac{1}{x} + b\right)^{-cx - 1/3} \sqrt{\frac{\pi}{2}} e^{-x - d} \sqrt{x + d} %
        \left[1 + \frac{55}{72\left(x + d\right)} - \frac{10151}{10368}\left(x+d\right)^2\right] ,
        \label{func: polynomial_appro}
      \end{equation}
      where $a,b,c,d$ are just positive constants and $c = 0.96 < 1$. As a result, 
      the part $(1/x + b)^{-cx - 1/3}$ in Function \ref{func: polynomial_appro} is infinity
      when $x$ is large though the total function is approximated to $0$. Thus, I have to 
      explicitly assign the result to be $0$ instead of calculating it. Indeed, this error 
      is not easy to find since in most cases the results are not infinity. 
          
    \subsection{Speed of Computation}
      I did not check any accurate benchmarks in the following discussions and they 
      depend on the average time of the simulations.
      The most obvious solution is to use multicores to do the computation. However, 
      most library functions do not support run concurrently and only run on a single core. 
      For example, I need to do many integrations and the speed of integration is critical. 
      I write some simple functions to utilize four CPU cores at the same time when doing 
      integration. This gives me a huge performance improvement.

      Furthermore, there are some facts about the basic operations. For instance, add is 
      faster than multiplication which is faster than division. Multiplications and divisions 
      are not associative between floating points. Though the performance differences between 
      different operations for integers can usually be optimized away by modern compilers, 
      the compilers can do nothing for floating points. Thus, I have to do it by ourselves. 
      For example, I have $z^2 - h_2\left(x\right)^2$ in function \ref{eq: twolayer_field}.
      In this formula, there are multiplications and one subtraction. After re-writing it to 
      $\left(z-h_2\right)\left(z + h_2\right)$, we have one addition, one subtraction and one 
      multiplication. Since addition and subtraction is not slower than multiplication, it 
      has no performance harm by the rewriting. What need to be noticed is that the 
      multiplication may not be slower than addition
      and it is dependent on machines. However, a division is definitely slower than the other three operations. 
      Therefore, in our program, expressions like $1 / 3$ are rewritten to $1*0.3333$ and so on.  

      Finally, since the program is written in C++, I use some new features of C++ to speed
      up the calculation. For instance, the keywords `constexpr' and `auto' is used a lot. 
      Meanwhile, I am really careful when allocate arrays trying to make the array fit into
      cache. 

  \section{Numerical Calculation of Spectra Based on the Two-Layer Model}
    After understanding the theory part of the two-layer model, 
    I then carried out numerical calculations of the spectra for the three MSPs
    based on the theory.  

    There are three independent parameters in the calculation. 
    The first parameter is fractional gap size $f=h_{2}/R_{lc}$, where $h_{2}$ is the 
    total gap size including both the primary acceleration region and the screening region 
    and $R_{lc}$ is the radius of the light cylinder. The second parameter is $g_{1}$ so that
    the charge density in the primary accelerating region is $\left(1-g_{1}\right) \rho_{GJ}$, 
    where $\rho_{GJ}$ is the Goldreich-Julian charge density. The third parameter is the ratio 
    between the sizes of the two gaps ($h_{1}/h_{2}$). Note that I only need to set the charge 
    densities in the primary acceleration region as an independent parameter, since the charge 
    densities in the two gaps are related to each other. Figures 
    \ref{fig: j0218_twolayer_cur.png}, \ref{fig: b1821_twolayer_cur.png}, and 
    \ref{fig: j1939_twolayer_cur.png} are the spectra of the three MSPs generated from 
    the two-layer model and the results of the 
    fit parameters are listed in Table \ref{table: twolayer_fit_parameter}. Other than the 
    low energy and high energy gamma-ray bands, the model is consistent with the observation 
    data in terms of the gamma-ray part. 

    % Generally speaking, the modeled spectra for the three MSPs are acceptable. Just as 
    % we have discussed in the data analysis part that global fits are not very consistent 
    % with separate fits in low energy gamma-ray band (about $100\mbox{MeV} - 500\mbox{MeV}$) 
    % and high energy band (above $10 \mbox{GeV}$), the modeled spectra also have the same 
    % problem. This can have two explanations. Firstly, the Fermi telescope is not sensitive 
    % below about $100 \mbox{MeV}$. As a result, the observational data may not be very precise
    % at this energy band. Secondly, the real emission mechanism in the energy band is different 
    % from the model describes. Thus, the obvious inconsistencies between the calculations 
    % and observations can be observed.

    \singleFig{j0218_twolayer_cur.png}{0.37}{The observed gamma-ray spectrum of PSR 
      J0218+4232 compared with the two-layer model.}
    \singleFig{b1821_twolayer_cur.png}{0.37}{The gamma-ray spectrum of PSR B1821$-$24 by 
      the two-layer model.}
    \singleFig{j1939_twolayer_cur.png}{0.34}{The gamma-ray spectrum of PSR B1937+21 by 
      the two-layer model. (blue line)}

        % \mayChange{I wanted to combine the three figures together, if I do so, each 
        %   figure is too small.}
        
    \begin{table}[!htp]
      \centering
      \scalebox{0.9}{
      \begin{tabular}{|cccc|}
        \hline
        Pulsar Name & $f$ & $g_1$ & $h_1/h_2$ \\ 
        \hline \hline
        J0218+4232 & 0.330 & 0.920 & 0.915 \\
        B1821$-$24 & 0.247 & 0.955 & 0.920 \\
        B1937+21 & 0.320 & 0.975 & 0.925 \\
        \hline 
      \end{tabular}}
      \caption[The results of fit parameters for the three MSPs.]
        {The results of fit parameters for the three MSPs. The physical 
        meaning of each parameter is consistent with the two-layer model described above.}
        % \change{Table is ugly, but not sure how to make it more beautiful...}}
      \label{table: twolayer_fit_parameter}
    \end{table}
    \vspace{0.5cm}
          
    After obtaining the spectral fit results in the gamma-ray band, I then generated 
    broadband spectra as Figures \ref{fig: j0218_twolayer_all.png}, 
    \ref{fig: b1821_twolayer_all.png}, and \ref{fig: j1939_twolayer_all_ave.png} show. 
    The hard X-ray data are from \cite{0004-637X-845-2-159}. By tweaking the independent 
    parameters of the two-layer model, I can make the modeled spectra 
    very close to the observational data in the hard X-ray bands. Since the lack of data in 
    the energy band from about $100$keV to $100$MeV, it is hard to tell if the two-layer model 
    describes the right physical scenario in this energy range. However, the prediction made 
    by the simplified two-layer model is generally accurate. In addition, the model is very 
    intuitive, which is also a very important consideration for building a model. Just as 
    the famous words "With four parameters I can fit an elephant, and with five I 
    can make him wiggle his trunk" 
    said by John von Neumann, in principle, we can fit any data by adding independent 
    parameters. Therefore, in order to test if a theoretical model is good or not, not only we 
    need to consider how precisely the model can predict, but also the physical meaning 
    behind the model. In this sense, the two-layer model is a good start of explaining 
    emission mechanism of pulsars. 
    
    \begin{figure}[!htp]
      \centering 
      \includegraphics[scale=0.37]{j0218_twolayer_all.png}
      \caption[The broadband spectrum of PSR J0218+4232.]
        {The broadband spectrum of PSR J0218+4232.
        The grey shade is the error of the global fit. And the green shade in the left panel of 
        the figure represents the error of hard X-ray spectrum. The `Total' legend represents the 
        total flux combining the synchrotron radiation, inverse-Compton radiation and curvature 
        radiation altogether.}
      \label{fig: j0218_twolayer_all.png}
    \end{figure}

    % \singleFig{j0218_twolayer_all.png}{0.37}{The broadband spectrum of PSR J0218+4232.
    %   The grey shade is the error of the global fit. And the green shade in the left panel of 
    %   the figure represents the error of hard X-ray. The 'Total' legend represents the total 
    %   flux combining the Synchrotron radiation, inverse Compton radiation and curvature 
    %   radiation altogether.}
    \vspace{0.5cm} 
    \begin{figure}[!htp]
      \centering 
      \includegraphics[scale=0.37]{b1821_twolayer_all.png}
      \caption[Broadband spectrum of PSR B1821$-$24.]
        {The broadband spectrum of PSR B1821$-$24. The meanings of grey shade and the green 
        shade are the same as Figure \ref{fig: j0218_twolayer_all.png}.}
      \label{fig: b1821_twolayer_all.png}
    \end{figure}

    % \singleFig{b1821_twolayer_all.png}{0.37}{The broadband spectrum of PSR B1821$-$24.
    %   The meanings of grey shade and the green shade are the same as Figure 
    %   \ref{fig: j0218_twolayer_all.png}.}
    \vspace{0.5cm} 
      
    \begin{figure}[!htp]
      \centering 
      \includegraphics[scale=0.37]{j1939_twolayer_all_ave.png}
      \caption[The broadband spectrum of PSR B1937+21.]
        {The broadband spectrum of PSR B1937+21. The meanings of grey shade and the green 
        shade are the same as Figure \ref{fig: j0218_twolayer_all.png}.}
      \label{fig: j1939_twolayer_all_ave.png}
    \end{figure}

    % \singleFig{j1939_twolayer_all_ave.png}{0.39}{The broadband spectrum of PSR B1937+21.
    %   The meanings of grey shade and the green shade are the same as Figure
    %   \ref{fig: j0218_twolayer_all.png}}
    \vspace{0.5cm}        

  

\chapter{Discussion}
  The parameters $f$, $g_1$, and $h_1/h_2$ have different impacts on the shapes of 
  the gamma-ray spectra as Figures \ref{fig: j0218_test_f}, \ref{fig: j0218_test_g1}, and 
  \ref{fig: j0218_test_ratio} show. The parameter $f$ controls the flux and cutoff energy
  ($\text{E}_{\text{cut}}$) of the spectra. Both the flux and $\text{E}_{\text{cut}}$ increase 
  with the increases of $f$. Moreover, this parameter has the most significant impacts on 
  the shape of the spectra. 
  The parameter $g_1$ mainly affects $\text{E}_{\text{cut}}$.
  Additionally, when $g_1$ is large enough (larger than $0.98$), the shapes of the spectra 
  becomes a little bit irregular. 
  The parameter which has the strangest influences on the spectra is $h_1/h_2$. As Figure
  \ref{fig: j0218_test_ratio} shows, the spectra are classified into two groups with 
  different cutoff energies. In each group, $h_1/h_2$ has nearly no impact on 
  $\text{E}_{\text{cut}}$. In addition, it also has small effects on the slopes of the 
  spectra below $\text{E}_{\text{cut}}$. 
  Like $g_1$, if $h_1/h_2$ becomes too large, the shapes of the spectra are a little 
  bit irregular. I have not found a convincing explanation yet and not sure if it reveals 
  some emission properties or just numerical errors when the parameters become too extreme. 

  The patterns discussed above only apply to a particular pulsar. For instance, 
  the values of $f$ of PSRs J0218+4232 and B1821$-$24 are $0.330$ and $0.247$.
  Thus $\text{E}_{\text{cut}}$ of PSR J0218+4232 should be larger than PSR B1821$-$24 because 
  $f$ has the most significant effects on the shapes of the spectra among the three parameters.  
  However, $\text{E}_{\text{cut}}$ is $3.77\pm0.40$ GeV for PSR J0218+4232 and 
  $4.5\pm0.71$ GeV for PSR B1821$-$24, which is not consistent with the 
  previous discussions. In addition, other pulsars does not follow the pattern neither as 
  Table \ref{table: two_layer_other_parameters} shows. This may be because that the spectral 
  properties are related to the sizes of light cylinders, magnetic fields at the light 
  cylinders, inclination angles and many other properties of pulsars. The three parameters 
  of the two-layer model describe the structure of the outer gap of a particular pulsar. 
  % In addition, the two-layer model used in the thesis is oversimplified,
  Therefore, we may not get much information by comparing the three parameters between 
  different pulsars. 

  \begin{figure}[!htp]
    \centering 
    \includegraphics[scale=0.45]{j0218_test_f.png}
    \caption[Effects of $f$ on the shape of the gamma-ray spectrum.]
      {Effects of $f$ on the shape of the gamma-ray spectrum. The parameters of PSR 
      J0218+4232 are used. The other two parameters are $g_1=0.92$ and $h_1/h_2=0.915$. }
    \label{fig: j0218_test_f}
  \end{figure}

  \begin{figure}[!htp]
    \centering 
    \includegraphics[scale=0.45]{j0218_test_g1.png}
    \caption[Effects of $g_1$ on the shape of the gamma-ray spectrum.]
      {Effects of $g_1$ on the shape of the gamma-ray spectrum. The parameters of PSR 
      J0218+4232 are used and $f=0.33$, $h_1/h_2=0.915$.}
    \label{fig: j0218_test_g1}
  \end{figure}

  \begin{figure}[!htp]
    \centering 
    \includegraphics[scale=0.45]{j0218_test_ratio.png}
    \caption[Effects of $h_1/h_2$ on the shape of the gamma-ray spectrum.]
      {Effects of $h_1/h_2$ on the shape of the gamma-ray spectrum. The parameters of PSR 
      J0218+4232 are used and $f=0.33$, $g_1=0.92$.}
    \label{fig: j0218_test_ratio}
  \end{figure}

  \begin{table}[!htp]
    \centering 
    \scalebox{0.9}{
      \begin{tabular}{|ccccccc|}
        \hline 
        Pulsar Name & $P$ (ms) & $f$ & $g_1$ & $h_1/h_2$ & $\makecell{E_{\text{cut}}\\ (\text{GeV})}$ & \makecell{Energy Flux  \\ ($10^{-11}$ erg $\text{cm}^{-2}$ $\text{s}^{-1}$)} \\
        \hline \hline 
        J0633+1746 & $237$ & $0.76$ & $0.85$ & $0.933$ & $2.2\pm0.1$ & $423.3\pm1.2$ \\ 
        J0835-4510 & $89.3$ & $0.16$ & $0.92$ & $0.927$ & $3.0\pm0.1$ & $906\pm2$ \\ 
        J0007+7303 & $316$ & $0.65$ & $0.94$ & $0.967$ & $4.7\pm0.2$ & $40\pm0.4$ \\ 
        J1057-5226 & $197$ & $0.60$ & $0.85$ & $0.933$ & $1.4\pm0.1$ & $29.5\pm0.3$ \\
        J0030+0451 & $4.9$ & $0.60$ & $0.88$ & $0.947$ & $1.8\pm0.2$ & $6.14\pm0.18$ \\ 
        J0437-4715 & $5.8$ & $0.38$ & $0.95$ & $0.927$ & $1.1\pm0.3$ & $1.67\pm0.11$ \\
        \hline
      \end{tabular}}
      \caption[The best-fit parameters of the two-layer model and observational properties 
        for a few pulsars]
        {The best-fit parameters of the two-layer model and observational properties 
        for a few pulsars. Both young pulsars and MSPs are included. }
      \label{table: two_layer_other_parameters}
  \end{table}
  \section{Constraints of the Two-Layer Model} 
    The simplified two-layer model is consistent with observational data to some extent. 
    (The relevant data are reported by \cite{0004-637X-720-1-178})
    The model uses four parameters to get a fairly good prediction of the gamma-ray 
    spectra of many pulsars. And all these four parameters have very obvious physical 
    meanings. However, the problems of the model are clear --- it is somewhat 
    oversimplified. Although there are other more sophisticated versions of the 
    two-layer model such as three-dimensional two-layer model 
    \cite{doi:10.1111/j.1365-2966.2011.18577.x} and I used the simpler 
    one, which may cause some inconsistencies between the simulations and observations. 

    Therefore, we can briefly analyze which part is oversimplified and can be improved. 
    First of all, we directly use a step function to describe the charged particle 
    distribution. Though the charge density of the screening region is much larger than 
    the primary region, using a step function is not very physical and may exaggerate 
    the change rate of charge density. At the same time, the dramatic change of charge 
    density also brings some numerical instabilities.

    Secondly, the model sets the total of screening region and primary region to be 
    rectangular. Though the actual shape is not clear, it should not be a 
    rectangular in theory and may be very different. In the numerical simulation, changes 
    in the shape of the regions will directly lead to a different integration region, 
    which may change the simulated spectra.
  
    Thirdly, there are some inconsistencies in the model itself according to its 
    assumptions. According to Equation \ref{eq: curvature_radiation_power} and Equation 
    \ref{eq: gamma_can_be_zero}, since $E_{\parallel}^{\prime}$ can be $0$, we know that 
    $\gamma_{e}$ can also be $0$, which is absolutely non-physical. Although this may not 
    have large influences on the spectra, it is the problem that should be avoided.

    All in all, though the model has some constraints, it is very physical and the gamma-ray 
    spectra computed based on the model is generally consistent with the observational data.

    % Those models may have addressed the problems described above, but the model used in 
    % the thesis do have some defects.  

  \section{Inconsistency Between the Two-Layer \\ Model and Fit Results}
    As Figures \ref{fig: j0218_twolayer_all.png}, \ref{fig: j1939_twolayer_all_ave.png}, and 
    \ref{fig: b1821_twolayer_all.png} show, the two-layer model does not fit well for all 
    three MSPs in the lower gamma-ray band (about $100$ MeV), even though
    I have tried different reasonable parameter combinations.  
    I think it is because the two-layer model used in the thesis is oversimplified. In 
    most cases, the spectrum produced by the theoretical model is not monotonic and highly 
    curved. However, the global fits of the PLEC model are usually very flat from $100$MeV 
    to $500$MeV. This means that I can hardly reproduce the similar shape in this energy 
    range, no matter what parameters I used. The fact that the gamma-ray emission predicted
    by the two-layer model is not as strong as the observational data in the energy range 
    from $100$MeV to $500$MeV may also be explained by inverse-Compton radiation. 

    % This could be attributed to a few reasons. 
    % First of all, 
    % the energy resolution of \textit{Fermi} LAT in a few hundred MeV is not as good as in 
    % other gamma-ray bands. Secondly, the energy band of several hundred MeV, the 
    % signal-to-noise is low compared with high energy part. By directly analyzing the 
    % events files, I find that about $90\%$ of the photons are between $100$MeV and $1000$MeV 
    % and the background is too bright compared with the target sources. 
    
    
   

    % Generally speaking, the modeled spectra for the three MSPs are acceptable. Just as 
    % we have discussed in the data analysis part that global fits are not very consistent 
    % with separate fits in low energy gamma-ray band (about $100\mbox{MeV} - 500\mbox{MeV}$) 
    % and high energy band (above $10 \mbox{GeV}$), the modeled spectra also have the same 
    % problem. This can have two explanations. Firstly, the Fermi telescope is not sensitive 
    % below about $100 \mbox{MeV}$. As a result, the observational data may not be very precise
    % at this energy band. Secondly, the real emission mechanism in the energy band is different 
    % from the model describes. Thus, the obvious inconsistencies between the calculations 
    % and observations can be observed.
  
\chapter{Conclusion and Future Works}
  \section{Conclusion}
    I do phase-averaged gamma-ray spectral analysis with about nine-year \textit{Fermi} LAT
    data for PSRs J0218+4232, B1821$-$24 and B1937+21. The fit results are listed in Tables 
    \ref{table: final_results} and \ref{table: final_results_j1939_pl}. The new results 
    have smaller error bars thanks to Pass 8 dataset and much more observational data.
    Meanwhile, the TS values for PSRs J0218+4232 and B1821$-$24 are larger than previous 
    studies as Table \ref{table: final_results} shows. For PSR B1937+21, the power-law 
    model is preferred since $TS_{cut} = 2\Delta log(likelihood) = -2.86 < 9$ and I follow 
    the convention in \cite{0067-0049-208-2-17} like the previous study
    \cite{0004-637X-787-2-167}.

    I also use the two-layer model to generate gamma-ray spectra for the three MSPs and obtain 
    broadband spectra by combining inverse-Compton scattering and synchrotron radiation.
    The broadband spectra for the three MSPs are generally consistent with the observation 
    in both hard X-rays and gamma-rays, except in a few hundred MeV energy range. Thus, it 
    is reasonable to speculate that the radio and gamma-ray emission regions are co-located. 

  \section{Analysis With LAT 8-year Point Source List}
    The latest preliminary LAT 8-year Point Source List (FL8Y) is released 
    on May 03, 2018. Since the release date is a little bit late and the point source list 
    is a preliminary version, I have not finished all the spectral analysis with the data.
    However, I would like to show some results I have done with the new source list
    together with my further plan.

    Some spectra models are changed, for instance, the expression of PLSuperExpCutoff model 
    has been changed as Equation \ref{eq: new_fermi_model} shows \footnote[1]%
    {\url{https://fermi.gsfc.nasa.gov/ssc/data/access/lat/fl8y/FL8Y_description_v8.pdf}}%\cite{newFermiModel}.
    \begin{equation}
      \frac{dN}{dE} = K\left(\frac{E}{E_0}\right)^{\Gamma} e^{a\left(E_0^b-E^b\right)} . 
      \label{eq: new_fermi_model}
    \end{equation}
    Particularly, Fermi tools combine the $E_0^b$ with $K$ as Equation 
    \ref{eq: new_fermi_model_modified} shows and the model is renamed to PLSuperExpCutoff2. 
    \begin{equation}
      \frac{dN}{dE} = K\left(\frac{E}{E_0}\right)^{\Gamma} e^{-aE^b} .
      \label{eq: new_fermi_model_modified}
    \end{equation}
    In principle, there is no difference between the Equations \ref{eq: fermi_model} and 
    \ref{eq: new_fermi_model_modified}, however, the parameters needed to be fitted are 
    different. The new point source list contains the source PSR B1937+21 and the default 
    spectra model is PLEC2. However, the fit results for the MSP are still not easy 
    to obtain. Table \ref{table: new_results_all_three} lists the fit parameters for all 
    the three MSPs.
    % \begin{table}[!htp]
    %   \centering
    %     \scalebox{0.8}{
    %     \begin{tabular}{|cccc|}
    %       \hline
    %       & J0218+4232 & B1821$-$24 & B1937+21 \\
    %       \hline \hline 
    %       Photon Index ($\Gamma$) & $1.77\pm0.07$ & $1.14\pm0.02$ & $1.84\pm0.03$ \\
    %       Expfactor (a, $10^{-3}$) & $6.73\pm0.86$ & $11.76\pm0.12$ & $6.77\pm0.25$ \\
    %       Scale ($E_0$) & $821.48$ & $1128.68$ & $1901$ \\ 
    %       Index2 (b) & $0.67$ & $0.67$ & $0.67$ \\
    %       Photon Flux ($10^{-8}$ $cm^{-2} s^{-1}$) & $7.44\pm0.32$ & $2.44\pm0.08$ & $3.00\pm0.19$ \\
    %       Energy Flux ($10^{-11}$ erg $cm^{-2} s^{-1}$) & $4.45\pm0.16$ & $2.08\pm0.04$ & $1.69\pm0.06$ \\
    %       TS value & $7189$ &  $980$  & $149$ \\
    %       \hline
    %     \end{tabular}}  
    %     \mycaption{Fit results of PSRs J0218+4232, B1821$-$24 and B1937+21 with 
    %       LAT 8-year Point Source List.}
    %     \label{table: new_results_all_three}        
    % \end{table}  

    \begin{table}[!htp]
      \centering
        \scalebox{0.8}{
        \begin{tabular}{|ccc|}
          \hline
          & J0218+4232 & B1821$-$24  \\
          \hline \hline 
          Photon Index ($\Gamma$) & $1.77\pm0.07$ & $1.14\pm0.02$  \\
          Expfactor (a, $10^{-3}$) & $6.73\pm0.86$ & $11.76\pm0.12$  \\
          Scale ($E_0$) & $821.48$ & $1128.68$  \\ 
          Index2 (b) & $0.67$ & $0.67$  \\
          Photon Flux ($10^{-8}$ $\text{cm}^{-2} \text{s}^{-1}$) & $7.44\pm0.32$ & $2.44\pm0.08$ \\
          Energy Flux ($10^{-11}$ erg $\text{cm}^{-2} \text{s}^{-1}$) & $4.45\pm0.16$ & $2.08\pm0.04$  \\
          TS value & $7189$ &  $980$  \\
          \hline
        \end{tabular}}  
        \mycaption{Fit results of PSRs J0218+4232 and B1821$-$24 with LAT 8-year Point 
          Source List.}
        \label{table: new_results_all_three}        
    \end{table}  
  
    The TS values for PSRs J0218+4232 and B1821$-$24 are both a little bit larger as Table 
    \ref{table: new_ts_compare} shows. Figures \ref{fig: j0218_new_cur} and 
    \ref{fig: b1821_new_cur} are the fit spectra of the PSRs
    J0218+4232 and B1821$-$24. I still have not got a good fit result for PSR B1937+21 because 
    this MSP takes more effort which is the same case as the 3FGL. Besides, the fit 
    results for the two MSPs are also not the final results because I only have tried a 
    few spectra models and have not compared the likelihood and TS values between different 
    models. 

    Figures \ref{fig: new_j0218_count_spectra} and \ref{fig: new_b1821_count_spectra} are 
    count spectra and count residuals plot for the two MSPs. 
    \begin{table}[!htp]
      \centering
        \scalebox{0.8}{
        \begin{tabular}{|ccc|}
          \hline
          & J0218+4232 & B1821$-$24\\
          \hline \hline 
          FL8Y & $7189$ & $980$  \\ 
          3FGL & $6809$ & $941$ \\ 
          \hline
        \end{tabular}}  
        \mycaption{TS values comparison between 3FGL (older) and FL8Y (newer) source list.}
        \label{table: new_ts_compare}        
    \end{table}  
    

    \begin{figure}[!htp]
      \centering 
      \includegraphics[scale=0.4]{j0218_new_cur.png}
      \caption{Gamma-ray spectra of PSR J0218+4232 with PLEC2 model.}
      \label{fig: j0218_new_cur}
    \end{figure}

    \begin{figure}[!htp]
      \centering 
      \includegraphics[scale=0.4]{b1821_new_cur.png}
      \caption{Gamma-ray spectra of PSR B1821$-$24 with PLEC2 model.}
      \label{fig: b1821_new_cur}
    \end{figure}

    %% add figure for j1939 

    \begin{figure}[!htp]
      \centering 
      \includegraphics[scale=0.36]{new_j0218_count_spectra.png}
      \caption{Count spectra and residuals of PSR J0218+4232}
      \label{fig: new_j0218_count_spectra}
    \end{figure}

    \begin{figure}[!htp]
      \centering 
      \includegraphics[scale=0.36]{new_b1821_count_spectra.png}
      \caption{Count spectra and residuals of PSR B1821$-$24.}
      \label{fig: new_b1821_count_spectra}
    \end{figure}
    

    \section{Future Works}
      LAT 8-year Point Source List is only a preliminary list and 4FGL is being processed. 
      I will continue to analyze the gamma-ray spectra of the three MSPs with the 
      newer point source list in order to get more precise results. In addition, since 
      \textit{Fermi} LAT is producing large quantities of data, it is important to be able 
      to handle the data effectively. I have created some pipeline scripts and am going to
      improve them, which can help me to study more pulsars with higher efficiency.
      Additionally, I am going to study the two-layer model more thoroughly and other 
      emission models, then compare the predictions of the models with newer observational data.


    % \begin{thebibliography}{9}


% \end{thebibliography}

% setting for reference 
\printbibliography
 
\end{document}





